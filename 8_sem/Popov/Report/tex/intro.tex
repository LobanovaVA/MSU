\section{Введение}

\subsection{Постановка задачи}

В работе будет рассматриваться разностная схема с центральными разностями ($\ln{\rho}$, $u$) для решения начально-краевых задач для системы уравнений, описывающей нестационарное двумерное движение вязкого баротропного газа:

\begin{equation} 
\label{eq:task}
\begin{dcases}
    \deriv{\rho}{t} + \div (\rho \vec{u}) = \rho f_0; \\
    \rho \bigg[ \deriv{\vec{u}}{t} + (\vec{u}, \, \nabla) \vec{u} \bigg] + \nabla p = L \vec{u} + \rho \vec{f}; \\
    L \vec{u} = \div (\mu \nabla \vec{u}) + \frac{1}{3} \nabla (\mu \div \vec{u}); \\
    p = p (\rho).
\end{dcases}
\end{equation}

\bigskip
В данной схеме известными считаются:
\begin{itemize}
	\item $\mu$ -- коэффициент вязкости газа, который считаем известной поло­жительной константой.
	\item $p$ -- функция давления газа.
	Будем использовать две возможные зависимости: \\
	$p(\rho) = C\rho$, где $C$ -- неотрицательная константа или \\
	$p(\rho) = \rho ^{\gamma}$, где $\gamma = 1.4$.
	\item $\vec{f} = \vec{f(x)}$ -- вектор внешних сил, который является функцией переменных Эйлера 
	$(t,\vec{x}) \in \Omega = \Omega_t \times \Omega_{\vec{x}} \subset \bR \times \bR^d$.
\end{itemize}

Неизвестными же считаются функции переменных Эйлера:
\begin{itemize}
	\item $\rho$ -- функция плотности;
	\item $\vec{u} = (u_1, \, \ldots, \, u_d)$ -- функция скорости.
\end{itemize}

Систему~\eqref{eq:task} можно переписать в виде
\begin{equation*} 
\label{eq:intermediate}
\begin{dcases}
    \deriv{\rho}{t} + \sum_{i = 1}^{d} \deriv{\rho u_i}{x_i} = \rho f_0; \\
    \deriv{\rho u_s}{t} + \sum_{i = 1}^{d} \deriv{\rho u_i u_s}{x_i} + \deriv{p}{x_s} = \mu \bigg( \sum_{i = 1}^{d} \deriv{^2 u_s}{x_i^2} + \frac{1}{3} \sum_{i = 1}^{d} \deriv{^2 u_i}{x_s \partial x_i} \bigg) + \rho f_s, & s = 1, \, \ldots, \, d.
\end{dcases}
\end{equation*}

Сделав замену $g = \ln \rho$ и ряд преобразований, систему~\eqref{eq:task} можно переписать в виде (см.~\cite{Popov})
\begin{equation} 
\label{eq:main}
\begin{dcases}
    \deriv{g}{t} + \frac{1}{2} \sum_{i = 1}^{d} \bigg( u_i \deriv{g}{x_i} + \deriv{u_i g}{x_i} + (2 - g) \deriv{u_i}{x_i} \bigg) = f_0; \\
    \begin{multlined}[b]
    \deriv{u_s}{t} + \frac{1}{3} \bigg( u_s \deriv{u_s}{x_s} + \deriv{u_s^2}{x_s} \bigg) + \frac{1}{2} \sum_{i = 1, \, i \not = s}^{d} \bigg( u_i \deriv{u_s}{x_i} + \deriv{u_i u_s}{x_i} - u_s \deriv{u_i}{x_i} \bigg) \, + \\ + p_\rho (e^g) \deriv{g}{x_s} = \mu e^g \bigg( \frac{4}{3} \deriv{^2 u_s}{x_s^2} + \sum_{i = 1, \, i \not = s}^{d} \bigg( \deriv{^2 u_s}{x_i^2} + \frac{1}{3} \deriv{^2 u_i}{x_s \partial x_i} \bigg) \bigg) + f_s,
    \end{multlined}
    & s = 1, \, \ldots, \, d.
\end{dcases}
\end{equation}

Дополним систему~\eqref{eq:task} начальными и граничными условиями:
\begin{equation} \label{eq:terms}
\begin{aligned}
    (\rho, \, \vec{u})|_{t = 0} = (\rho_0, \, \vec{u}_0), \quad \vec{x} \in \Omega_x; \\
    u (t, \, \vec{x}) = 0, \quad (t, \, \vec{x}) \in \Omega_t \times \partial \Omega_x.
\end{aligned}
\end{equation}

В качестве областей $\Omega_t$ и $\Omega_x$ рассмотрим $[0; \, T] \subset \bR$ и $\Omega_{x_1} \times \ldots \times \Omega_{x_d} \subset \bR^d$, где $\Omega_{x_s} = [0; \, X_s]$, $s = 1, \, \ldots, \, d$, соответственно.




\subsection{Основные обозначения}\label{sec:notation}
Введем на областях $\Omega_t = [0; T]$ и $\Omega_{x_s} = [0; X_s]$, 
равномерные сетки с шагом $\tau$ и $h_s$ соответственно:
$$ \omega_{\tau} = \{n \tau \,|\, n = 0,\dots,N\},   \;\text{где}\; N \tau = T  $$
$$ \omega_{h_s}  = \{m h_s  \,|\, m = 0,\dots,M_s\}, \;\text{где}\; M_s h_s = X_s $$

Обозначим 
$$ 
   \vec{h} = (h_1, \, \ldots, \, h_d), \quad
   \omega_{\vec{h}} = (\omega_{h_1} \times \, \ldots \times \, \omega_{h_d}), \quad
   \omega_{\tau, \, h} = \omega_{\tau} \times \omega_h
$$

$$ \gamma_{\vec{h}, \, s}^- = \omega_{h_1}     \times \ldots \times \omega_{h_{s-1}} 
        \times \{ 0 \} \times \omega_{h_{s+1}} \times \ldots \times \omega_{h_d}     $$

$$ \gamma_{\vec{h}, \, s}^+ = \omega_{h_1}     \times \ldots \times \omega_{h_{s-1}}
      \times \{ X_s \} \times \omega_{h_{s+1}} \times \ldots \times \omega_{h_d}    $$

$$ \gamma_{\vec{h}, \, s} = \gamma_{\vec{h}, \, s}^- \cup \gamma_{\vec{h}, \, s}^+, \quad
   \gamma_{\vec{h}} = \gamma_{\vec{h}, \, 1} \cup \ldots \cup \gamma_{\vec{h}, \, d}  $$ 


Для сокращения записи обозначим 
$m = (m_1, \, \ldots, \, m_d)$, $m \pm q_s = (m_1, \, \ldots, m_{s-1}, \, m_s \pm q, \, m_{s+1}, \, \ldots, \, m_d)$,
значение для произвольной функции $g$ в узле $(n, \, m)$ через $g^n_m$, $g^{n+1}_m$ через $\widehat{g}$. 

Введем обозначения для среднего значения величин сеточной функции в двух соседних узлах:
\begin{align*}
  g_{\avg_s}      = \frac{g^n_m + g^n_{m + 1_s}}{2}    \,,\quad
  g_{\overline{\avg}_s} = \frac{g^n_m + g^n_{m - 1_s}}{2}            
\end{align*}
и для разностных операторов:
\begin{align*}
    g_t = \frac{g^{n+1}_m - g^n_m}{\tau}&  \,,\quad
    g_{x_s} = \frac{g^n_{m + 1_s} - g^n_m}{h_s}, \\
    g_{\overline{x}_s} = \frac{g^n_m - g^n_{m - 1_s}}{h_s}&  \,,\quad 
    g_{\mathring{x}_s} = \frac{g^n_{m + 1_s} - g^n_{m - 1_s}}{2 h_s},
\end{align*}
\begin{equation*}
    g_{x_s \overline{x}_s} = (g_{x_s})_{\overline{x}_s} = \frac{g^n_{m + 1_s} - 2 g^n_m + g^n_{m - 1_s}}{h_s^2}.
\end{equation*}