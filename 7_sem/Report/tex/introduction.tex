\section{Введение}

\subsection{Постановка задачи}

В работе будет рассматриваться разностная схема с центральными разностями ($\ln{\rho}$, $u$) для решения начально-краевых задач для системы уравнений, описывающей нестационарное одномерное движение вязкого баротропного газа:

\begin{equation}
\label{system:1}
\left\{
 \begin{aligned}
  & \dfrac{\partial \rho}{\partial t} 
    + \dfrac{\partial \rho u}{\partial x} 
    = 0 \\
  & \rho \dfrac{\partial u}{\partial t} 
    + \rho u \dfrac{\partial u}{\partial x} 
    + \dfrac {\partial p} {\partial x} 
    = \mu \dfrac{\partial^{2} u}{\partial x ^{2}} 
    + \rho f \\
  & p = p (\rho)
 \end{aligned}
\right.
\end{equation}

\bigskip
В данной схеме известными считаются:
\begin{itemize}
	\item $\mu$ -- коэффициент вязкости газа, который считаем известной поло­жительной константой.
	\item $p$ -- функция давления газа.
	Будем использовать две возможные зависимости: \\
	$p(\rho) = C\rho$, где $C$ -- неотрицательная константа или \\
	$p(\rho) = \rho ^{\gamma}$, где $\gamma = 1.4$.
	\item $f$ -- вектор внешних сил, являющийся функцией переменных Эйлера 
	$(t,x) \in Q = [0,T] \times [0,X]$.
\end{itemize}

Неизвестными же считаются функции переменных Эйлера:
\begin{itemize}
	\item $\rho$ -- функция плотности;
	\item $u$ -- функция скорости.
\end{itemize}

Дополним систему (\ref{system:1}) начальными и граничными условиями.
В начальный момент времени задаются функции, значениями которых являются плотность и скорость газа в точках отрезка $[0, X]$, а граничными условиями являются условия непротекания.

\begin{equation}
\label{system:2}
\left\{
 \begin{aligned}
  & (\rho, u)|_{t = 0} =
    (\rho_{0}, \,u_{0}), \,  
    x \in [0, X] \\
  & u(t, 0) = u(t, X) = 0, \,
    t \in [0, T] 
 \end{aligned}
\right.
\end{equation}


\subsection{Преобразования}
Преобразуем систему (\ref{system:1}) следующим образом. Раскроем конвективное слагаемое, используя формулу дифференцирования произведения функций, поделим получившиеся выражение на $\rho$, обозначим $g = \ln \rho$ и получим:
$$
  \frac{1}{\rho} \dfrac{\partial \rho}{\partial t} 
  + \frac{u}{\rho} \dfrac{\partial \rho}{\partial x} 
  + \dfrac{\partial u}{\partial x} = 0 
  \;\Rightarrow\;
  \dfrac{\partial g}{\partial t} 
  + u \dfrac{\partial g}{\partial x} 
  + \dfrac{\partial u}{\partial x} = 0 
$$$$
  \dfrac{\partial g}{\partial t} 
  + \frac{1}{2} \left(
    u \dfrac{\partial g}{\partial x} 
    + \dfrac{\partial ug}{\partial x} 
    + (2 - g)\dfrac{\partial u}{\partial x}
  \right) = 0 
$$

Аналогичное проделаем со вторым уравнением:
$$
  \dfrac{\partial u}{\partial t} 
  + \frac{1}{3} \left(
    u \dfrac{\partial u}{\partial x} 
    + \dfrac{\partial u^2}{\partial x} 
  \right) 
  + \tilde{p}'(e^{g}) \dfrac{\partial g}{\partial x} 
  = \mu e^{-g} \dfrac{\partial^2 u}{\partial x^2} + f
$$
где  $\tilde{p}'(x) = \dfrac{\partial p}{\partial \rho}(x)$

В итоге система (\ref{system:1}) преобразуется к виду:
\begin{equation}
\label{system:3}
\left\{
 \begin{aligned}
  & \dfrac{\partial g}{\partial t} 
    + \frac{1}{2} \left(
      u \dfrac{\partial g}{\partial x} 
      + \dfrac{\partial ug}{\partial x} 
      + (2 - g)\dfrac{\partial u}{\partial x}
    \right) = f_0  \\
  & \dfrac{\partial u}{\partial t} 
    + \frac{1}{3} \left(
      u \dfrac{\partial u}{\partial x} 
      + \dfrac{\partial u^2}{\partial x} 
    \right) 
    + \tilde{p}'(e^{g}) \dfrac{\partial g}{\partial x} 
    = \mu e^{-g} \dfrac{\partial^2 u}{\partial x^2} + f \\
  & \tilde{p}'(x) = \dfrac{\partial p}{\partial \rho}(x) \\
  & g = \ln \rho \\
  & p = p (\rho)
 \end{aligned}
\right.
\end{equation}


\subsection{Основные обозначения}
Рассмотрим временной интервал $\Omega_t = [0,T]$ и пространственную область в виде отрезка $\Omega_x = [0, X]$, введем на них равномерные сетки с шагом $\tau$ и $h$ соответственно:
$$ \bar w_{\tau} = \{n \tau \,|\, n = 0,\dots,N\}, \;\text{где}\; N \tau = T $$
$$ \bar w_{h} = \{mh \,|\, m = 0,\dots,M\}, \;\text{где}\; Mh = X $$
В области $ Q := \Omega_{t} \times \Omega_{x} $ 
вводится сетка $ \bar Q_{\tau h} := \bar \omega_{\tau} \times \bar \omega_{h} $.


Значение функции $g$ в узле $(m, n)$ будем обозначать через $g_{n}^{m}$
Для сокращения записей будем также использовать следующие обозначения:\label{obozn}
$$ 
  g_{m}^{n+1} = \hat g 
  \,,\quad 
  g_{m \pm 1}^{n} = g^{\pm 1} 
$$

Обозначения для среднего значения величин сеточной функции в двух соседних узлах:
$$
  g_{s} = \frac{g_{m+1}^{n} + g_{m}^{n}}{2}
  \,,\quad
  g_{\bar s} = \frac {g_{m}^{n} + g_{m-1}^{n}}{2}
$$

Для разностных операторов применяются следующие обозначения:
$$
  g_{t}=\frac{g_{m}^{n+1}-g_{m}^{n}}{\tau}
  \,,\quad
  g_{x}=\frac{g_{m+1}^{n}-g_{m}^{n}}{h}
$$
$$
  g_{\mathring{x}} = \frac{g^n_{m + 1} - g^n_{m - 1}}{2 h}
  \,,\quad
  g_{\bar{x}}=\frac{g_{m}^{n}-g_{m-1}^{n}}{h}
$$
$$
  g_{x \bar{x}}= (g_{x})_{\bar{x}} 
  = \frac{g_{m-1}^{n}-2 g_{m}^{n}+g_{m+1}^{n}}{h^{2}}
$$
