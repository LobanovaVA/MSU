\section{Отладочный тест}

\subsection{Постановка задачи}
Рассмотрим область $Q = [0,1] \times[0, 1]$ и зададим функции $\rho$ и $u$:
\begin{align}
\label{debug_f}
& \tilde{\rho}(t, x) = e^{t} (\cos(3 \pi x) + 1.5) &
& \tilde{u}(t, x) = \cos(2 \pi t) \sin(4 \pi x)
\end{align}

Определим функции $f_{0}$ (отличную от нуля правую часть уравнения неразрывности) и $f$ так, чтобы функции $\tilde{\rho}$ и $\tilde{u}$ удовлетворяли системе (\ref{system:3}) с правой частью, составленной из этих функций, а именно:
\begin{equation*}
\left\{
 \begin{aligned}
  & \dfrac{\partial \tilde{g}}{\partial t} 
    + \tilde{u} \dfrac{\partial \tilde{g}} {\partial x} 
    + \dfrac{\partial \tilde{u}} {\partial x} 
    = f_0 \\
  & \dfrac{\partial \tilde{u}}{\partial t} 
    + \tilde{u} \dfrac{\partial \tilde{u}}{\partial x} 
    + \tilde{p}'(e^{g}) \dfrac{\partial g}{\partial x} 
    = \mu e^{-g} \dfrac{\partial^{2} \tilde{u}}{\partial x ^{2}} 
    + f \\
  & \tilde{p}'(x) = \dfrac{\partial p}{\partial \rho}(x) \\
  & p = p(\tilde{\rho})
 \end{aligned}
\right.
\end{equation*}

Выпишем отдельно все частные производные, необходимые для подсчета функций $f$ и $f_{0}$:
\begin{align*}
  \dfrac{\partial \tilde{\rho}}{\partial t}
    &= e^{t} (\cos(3 \pi x) + 1.5) = \tilde{\rho}
  & \dfrac{\partial \tilde{\rho}}{\partial x}
    &= - 3\pi e^{t} \sin(3 \pi x) \\
  \dfrac{\partial \tilde{g}}{\partial t} 
    &= \dfrac{\partial \ln \tilde{\rho}}{\partial t}
     = \frac{1}{\tilde{\rho}} \dfrac{\partial \tilde{\rho}}{\partial t} = 1 
  & \dfrac{\partial \tilde{g}}{\partial x} 
    &= \dfrac{\partial \ln \tilde{\rho}}{\partial x}
     = \frac{1}{\tilde{\rho}} \dfrac{\partial \tilde{\rho}}{\partial x} \\
  \dfrac{\partial \tilde{u}}{\partial t}
    &= -2\pi \sin(2 \pi t) \sin(4 \pi x) 
  & \dfrac{\partial \tilde{u}}{\partial x}
    &=  4\pi \cos(2 \pi t) \cos(4 \pi x)
\end{align*}
\begin{equation*}
  \dfrac{\partial^{2} \tilde{u}}{\partial x^2}
    = -16\pi^2 \cos(2 \pi t) \sin(4 \pi x)
\end{equation*}

Выписывать явный вид для функций $f$ и $f_{0}$ не имеет смысла, так они являются комбинациями описанных выше функций и производных и намного проще реализовать отдельные части этих комбинаций.

Таким образом, функции (\ref{debug_f}) являются гладким точным решением дифференциальной задачи (\ref{system:1}) с начальными и граничными условиями:
\begin{equation*}
\begin{aligned}
  & \tilde{\rho}(0,x) = \cos(3 \pi x) + 1.5, x \in [0,1] \\
  & \tilde{u}(0,x) = \sin(4 \pi x), x \in[0,1] \\
  & \tilde{u}(t,0) = u(t,1) = 0, t \in [0,1]
\end{aligned}
\end{equation*}

\subsection{Численные эксперименты}
Будем рассматривать зависимости
$$ 
  p(\rho) = C \rho, \; C \in \{1, 10, 100\} ; \quad
  p(\rho) = \rho^{\gamma}, \; \gamma = 1.4
$$ 

Ниже приведены результаты численных экспериментов, а именно таблицы с ошибками (нормами разности между разностным решением и точным решением дифференциальной задачи на последнем временном слое; нормы $C, L_{2}, W_{2}^{1}$) для различных пар параметра $\mu \in \{0.1, 0.01, 0.001\}$ и зависимости $p(\rho)$. \\

Для каждой такой пары также варьируются шаги сетки $\tau$ и $h$:
$$ (\tau, h) \in \{ 10^{-1}, 10^{-2}, 10^{-3}, 10^{-4} \}^2 $$ 

\newpage

\subsubsection{Ошибки для $G$}
\begin{table}[H]
\centering
\begin{tabular}{|c|c|c|c|c|}
\hline
\diagTH & $10^{-1}$ & $10^{-2}$ & $10^{-3}$ & $10^{-4}$ \\
\hline
 & \texttt{1.181652e+01} & \texttt{1.443212e+00} & \texttt{1.443281e+00} & \texttt{1.443604e+00} \\
$10^{-1}$ & \texttt{4.893319e+00} & \texttt{8.042684e-01} & \texttt{8.003429e-01} & \texttt{8.002513e-01} \\
 & \texttt{3.968220e+01} & \texttt{7.933857e+00} & \texttt{7.912020e+00} & \texttt{7.926450e+00} \\
\hline
 & \texttt{6.248864e+01} & \texttt{1.766651e-01} & \texttt{1.708897e-01} & \texttt{1.708306e-01} \\
$10^{-2}$ & \texttt{1.900438e+01} & \texttt{9.811932e-02} & \texttt{9.445940e-02} & \texttt{9.442351e-02} \\
 & \texttt{1.708589e+02} & \texttt{7.879295e-01} & \texttt{7.587913e-01} & \texttt{7.584994e-01} \\
\hline
 & \texttt{7.120057e+00} & \texttt{2.252087e-02} & \texttt{1.756171e-02} & \texttt{1.751225e-02} \\
$10^{-3}$ & \texttt{1.926111e+00} & \texttt{1.232548e-02} & \texttt{9.001833e-03} & \texttt{8.971483e-03} \\
 & \texttt{2.143837e+01} & \texttt{1.042810e-01} & \texttt{8.186950e-02} & \texttt{8.180130e-02} \\
\hline
 & \texttt{2.155197e+00} & \texttt{6.803025e-03} & \texttt{1.804239e-03} & \texttt{1.755320e-03} \\
$10^{-4}$ & \texttt{1.278056e+00} & \texttt{4.675248e-03} & \texttt{9.227605e-04} & \texttt{8.926073e-04} \\
 & \texttt{1.193935e+01} & \texttt{3.956212e-02} & \texttt{8.415822e-03} & \texttt{8.227393e-03} \\
\hline
\end{tabular}
\caption{Ошибки для схемы \texttt{1} при $C = 1$, $\gamma = 1$ и~$\mu = 0.1$.}
\end{table}

\begin{table}[H]
\centering
\begin{tabular}{|c|c|c|c|c|}
\hline
\diagTH & $10^{-1}$ & $10^{-2}$ & $10^{-3}$ & $10^{-4}$ \\
\hline
 & \texttt{3.222515e+00} & \texttt{1.462851e-01} & \texttt{1.440101e-01} & \texttt{1.439917e-01} \\
$10^{-1}$ & \texttt{1.120161e+00} & \texttt{6.801977e-02} & \texttt{6.715263e-02} & \texttt{6.714411e-02} \\
 & \texttt{1.339440e+01} & \texttt{8.824962e-01} & \texttt{9.035803e-01} & \texttt{9.042429e-01} \\
\hline
 & \texttt{5.048784e+01} & \texttt{2.294542e-02} & \texttt{2.236461e-02} & \texttt{2.235894e-02} \\
$10^{-2}$ & \texttt{1.562821e+01} & \texttt{1.154502e-02} & \texttt{1.111025e-02} & \texttt{1.110629e-02} \\
 & \texttt{9.405579e+01} & \texttt{1.030876e-01} & \texttt{9.997687e-02} & \texttt{1.001472e-01} \\
\hline
 & \texttt{9.549526e+02} & \texttt{3.727700e-03} & \texttt{2.617173e-03} & \texttt{2.610631e-03} \\
$10^{-3}$ & \texttt{4.195816e+02} & \texttt{1.869996e-03} & \texttt{1.296306e-03} & \texttt{1.292612e-03} \\
 & \texttt{1.360581e+03} & \texttt{2.412181e-02} & \texttt{1.041279e-02} & \texttt{1.037825e-02} \\
\hline
 & \texttt{8.625099e+02} & \texttt{3.212226e-03} & \texttt{2.719209e-04} & \texttt{2.652392e-04} \\
$10^{-4}$ & \texttt{4.645395e+02} & \texttt{9.819210e-04} & \texttt{1.352273e-04} & \texttt{1.313576e-04} \\
 & \texttt{1.326580e+03} & \texttt{1.802712e-02} & \texttt{1.133119e-03} & \texttt{1.042953e-03} \\
\hline
\end{tabular}
\caption{Ошибки для схемы \texttt{1} при $C = 10$, $\gamma = 1$ и~$\mu = 0.1$.}
\end{table}

\begin{table}[H]
\centering
\begin{tabular}{|c|c|c|c|c|}
\hline
\diagTH & $10^{-1}$ & $10^{-2}$ & $10^{-3}$ & $10^{-4}$ \\
\hline
 & \texttt{4.851430e+04} & \texttt{1.183144e-01} & \texttt{1.022894e-02} & \texttt{1.020089e-02} \\
$10^{-1}$ & \texttt{2.208050e+04} & \texttt{1.769615e-02} & \texttt{6.105316e-03} & \texttt{6.099189e-03} \\
 & \texttt{3.103531e+05} & \texttt{1.126021e+00} & \texttt{6.339364e-02} & \texttt{6.348438e-02} \\
\hline
 & \texttt{2.537852e+03} & \texttt{4.145018e-03} & \texttt{2.241746e-03} & \texttt{2.239338e-03} \\
$10^{-2}$ & \texttt{1.606647e+03} & \texttt{1.611664e-03} & \texttt{1.183370e-03} & \texttt{1.183675e-03} \\
 & \texttt{1.895135e+04} & \texttt{2.015021e-02} & \texttt{4.378571e-03} & \texttt{4.334379e-03} \\
\hline
 & \texttt{1.434099e+02} & \texttt{3.399383e-03} & \texttt{2.303365e-04} & \texttt{2.279940e-04} \\
$10^{-3}$ & \texttt{1.111249e+02} & \texttt{1.130865e-03} & \texttt{1.205005e-04} & \texttt{1.208826e-04} \\
 & \texttt{2.821377e+02} & \texttt{1.889347e-02} & \texttt{4.640649e-04} & \texttt{4.105267e-04} \\
\hline
 & \texttt{1.704815e+04} & \texttt{3.402526e-03} & \texttt{4.171373e-05} & \texttt{2.199472e-05} \\
$10^{-4}$ & \texttt{1.281007e+04} & \texttt{1.142852e-03} & \texttt{1.560524e-05} & \texttt{1.165764e-05} \\
 & \texttt{3.882601e+04} & \texttt{1.900494e-02} & \texttt{1.948088e-04} & \texttt{3.916520e-05} \\
\hline
\end{tabular}
\caption{Ошибки для схемы \texttt{1} при $C = 100$, $\gamma = 1$ и~$\mu = 0.1$.}
\end{table}

\begin{table}[H]
\centering
\begin{tabular}{|c|c|c|c|c|}
\hline
\diagTH & $10^{-1}$ & $10^{-2}$ & $10^{-3}$ & $10^{-4}$ \\
\hline
 & \texttt{9.118165e+00} & \texttt{1.153563e+00} & \texttt{1.162321e+00} & \texttt{1.162278e+00} \\
$10^{-1}$ & \texttt{3.344701e+00} & \texttt{6.000072e-01} & \texttt{5.989285e-01} & \texttt{5.990541e-01} \\
 & \texttt{3.390989e+01} & \texttt{6.012765e+00} & \texttt{5.659256e+00} & \texttt{5.685288e+00} \\
\hline
 & \texttt{8.146566e+00} & \texttt{9.039186e-02} & \texttt{8.840905e-02} & \texttt{8.838949e-02} \\
$10^{-2}$ & \texttt{5.449893e+00} & \texttt{5.340469e-02} & \texttt{5.216930e-02} & \texttt{5.215748e-02} \\
 & \texttt{3.567165e+01} & \texttt{4.388565e-01} & \texttt{4.531820e-01} & \texttt{4.561369e-01} \\
\hline
 & \texttt{3.864207e+00} & \texttt{1.062531e-02} & \texttt{8.764154e-03} & \texttt{8.747440e-03} \\
$10^{-3}$ & \texttt{1.116353e+00} & \texttt{6.322317e-03} & \texttt{4.838855e-03} & \texttt{4.828481e-03} \\
 & \texttt{1.053306e+01} & \texttt{4.526193e-02} & \texttt{4.443798e-02} & \texttt{4.485572e-02} \\
\hline
 & \texttt{4.602507e+00} & \texttt{7.991192e-03} & \texttt{8.912093e-04} & \texttt{8.744621e-04} \\
$10^{-4}$ & \texttt{1.249849e+00} & \texttt{2.789505e-03} & \texttt{4.899526e-04} & \texttt{4.791004e-04} \\
 & \texttt{1.250676e+01} & \texttt{2.929977e-02} & \texttt{4.359942e-03} & \texttt{4.469110e-03} \\
\hline
\end{tabular}
\caption{Ошибки для схемы \texttt{1} при $C = 1$, $\gamma = 1.4$ и~$\mu = 0.1$.}
\end{table}

\begin{table}[H]
\centering
\begin{tabular}{|c|c|c|c|c|}
\hline
\diagTH & $10^{-1}$ & $10^{-2}$ & $10^{-3}$ & $10^{-4}$ \\
\hline
 & \texttt{4.564166e+00} & \texttt{9.494955e-01} & \texttt{9.486571e-01} & \texttt{9.486554e-01} \\
$10^{-1}$
 & \texttt{1.573963e+00} & \texttt{5.196175e-01} & \texttt{5.187592e-01} & \texttt{5.187529e-01} \\
 & \texttt{1.560271e+01} & \texttt{5.156837e+00} & \texttt{5.054622e+00} & \texttt{5.063952e+00} \\
\hline
 & \texttt{4.794683e+01} & \texttt{1.792192e-01} & \texttt{1.776189e-01} & \texttt{1.776132e-01} \\
$10^{-2}$
 & \texttt{1.094421e+01} & \texttt{8.201671e-02} & \texttt{8.096468e-02} & \texttt{8.095492e-02} \\
 & \texttt{1.408516e+02} & \texttt{1.382530e+00} & \texttt{1.425657e+00} & \texttt{1.431199e+00} \\
\hline
 & \texttt{1.167069e+04} & \texttt{2.340101e-02} & \texttt{2.252434e-02} & \texttt{2.251531e-02} \\
$10^{-3}$
 & \texttt{5.143610e+03} & \texttt{1.035069e-02} & \texttt{8.646357e-03} & \texttt{8.636329e-03} \\
 & \texttt{1.591441e+04} & \texttt{1.650773e-01} & \texttt{1.838053e-01} & \texttt{1.841455e-01} \\
\hline
 & \texttt{3.367458e+04} & \texttt{5.953102e-03} & \texttt{2.323615e-03} & \texttt{2.314594e-03} \\
$10^{-4}$
 & \texttt{1.547585e+04} & \texttt{4.247154e-03} & \texttt{8.816500e-04} & \texttt{8.710264e-04} \\
 & \texttt{5.749273e+04} & \texttt{4.241223e-02} & \texttt{1.887922e-02} & \texttt{1.910149e-02} \\
\hline
\end{tabular}
\caption{Ошибки для схемы \texttt{1} при $C = 1$, $\gamma = 1$ и~$\mu = 0.01$.}
\end{table}

\begin{table}[H]
\centering
\begin{tabular}{|c|c|c|c|c|}
\hline
\diagTH & $10^{-1}$ & $10^{-2}$ & $10^{-3}$ & $10^{-4}$ \\
\hline
 & \texttt{2.949404e+01} & \texttt{1.407143e-01} & \texttt{1.364428e-01} & \texttt{1.364126e-01} \\
$10^{-1}$
 & \texttt{1.364401e+01} & \texttt{6.651169e-02} & \texttt{6.472063e-02} & \texttt{6.471224e-02} \\
 & \texttt{7.068992e+01} & \texttt{1.374316e+00} & \texttt{8.260481e-01} & \texttt{8.278018e-01} \\
\hline
 & \texttt{1.481902e+02} & \texttt{3.883304e-02} & \texttt{3.797993e-02} & \texttt{3.797133e-02} \\
$10^{-2}$
 & \texttt{1.112713e+02} & \texttt{2.198733e-02} & \texttt{2.132080e-02} & \texttt{2.131430e-02} \\
 & \texttt{5.665088e+02} & \texttt{2.085437e-01} & \texttt{2.067382e-01} & \texttt{2.070684e-01} \\
\hline
 & \texttt{9.670434e+02} & \texttt{5.794690e-03} & \texttt{4.723141e-03} & \texttt{4.712316e-03} \\
$10^{-3}$
 & \texttt{6.789979e+02} & \texttt{3.066431e-03} & \texttt{2.658904e-03} & \texttt{2.656186e-03} \\
 & \texttt{1.396412e+03} & \texttt{3.131827e-02} & \texttt{2.164434e-02} & \texttt{2.161422e-02} \\
\hline
 & \texttt{9.197238e+04} & \texttt{2.514426e-03} & \texttt{4.946716e-04} & \texttt{4.827475e-04} \\
$10^{-4}$
 & \texttt{7.615748e+04} & \texttt{1.369688e-03} & \texttt{2.735480e-04} & \texttt{2.726988e-04} \\
 & \texttt{1.859797e+05} & \texttt{1.963779e-02} & \texttt{2.236631e-03} & \texttt{2.189775e-03} \\
\hline
\end{tabular}
\caption{Ошибки для схемы \texttt{1} при $C = 10$, $\gamma = 1$ и~$\mu = 0.01$.}
\end{table}

\begin{table}[H]
\centering
\begin{tabular}{|c|c|c|c|c|}
\hline
\diagTH & $10^{-1}$ & $10^{-2}$ & $10^{-3}$ & $10^{-4}$ \\
\hline
 & \texttt{2.464089e+03} & \texttt{4.004913e-01} & \texttt{1.236771e-02} & \texttt{1.236528e-02} \\
$10^{-1}$
 & \texttt{9.520705e+02} & \texttt{9.438113e-02} & \texttt{6.455602e-03} & \texttt{6.451369e-03} \\
 & \texttt{1.004560e+04} & \texttt{8.056591e+00} & \texttt{5.645747e-02} & \texttt{5.615298e-02} \\
\hline
 & \texttt{2.047359e+03} & \texttt{3.790905e-01} & \texttt{1.282005e+06} & \texttt{7.053810e+03} \\
$10^{-2}$
 & \texttt{1.285078e+03} & \texttt{2.368079e-01} & \texttt{5.063466e+04} & \texttt{3.578959e+02} \\
 & \texttt{1.499430e+04} & \texttt{3.709077e+00} & \texttt{7.126936e+07} & \texttt{1.727106e+06} \\
\hline
 & \texttt{4.428032e+01} & \texttt{3.426404e-03} & \texttt{1.629036e-04} & \texttt{1.296869e-02} \\
$10^{-3}$
 & \texttt{1.006138e+01} & \texttt{1.137896e-03} & \texttt{8.384069e-05} & \texttt{1.427647e-04} \\
 & \texttt{1.359861e+02} & \texttt{1.892302e-02} & \texttt{3.215977e-04} & \texttt{8.021061e-01} \\
\hline
 & \texttt{6.494181e+02} & \texttt{3.638056e-03} & \texttt{3.630427e-05} & \texttt{1.024102e-05} \\
$10^{-4}$
 & \texttt{3.778433e+02} & \texttt{1.213182e-03} & \texttt{1.223737e-05} & \texttt{5.829744e-06} \\
 & \texttt{1.120882e+03} & \texttt{1.942868e-02} & \texttt{1.952241e-04} & \texttt{2.193910e-05} \\
\hline
\end{tabular}
\caption{Ошибки для схемы \texttt{1} при $C = 100$, $\gamma = 1$ и~$\mu = 0.01$.}
\end{table}

\begin{table}[H]
\centering
\begin{tabular}{|c|c|c|c|c|}
\hline
\diagTH & $10^{-1}$ & $10^{-2}$ & $10^{-3}$ & $10^{-4}$ \\
\hline
 & \texttt{1.081077e+01} & \texttt{1.405509e+00} & \texttt{1.554279e+00} & \texttt{1.530040e+00} \\
$10^{-1}$
 & \texttt{2.953611e+00} & \texttt{6.648733e-01} & \texttt{6.465171e-01} & \texttt{6.464135e-01} \\
 & \texttt{2.850794e+01} & \texttt{3.091198e+01} & \texttt{1.742743e+01} & \texttt{1.674844e+01} \\
\hline
 & \texttt{4.488869e+01} & \texttt{1.886602e-01} & \texttt{1.947784e-01} & \texttt{1.948402e-01} \\
$10^{-2}$
 & \texttt{1.032222e+01} & \texttt{9.125062e-02} & \texttt{9.149386e-02} & \texttt{9.149660e-02} \\
 & \texttt{1.392258e+02} & \texttt{8.469457e-01} & \texttt{8.583813e-01} & \texttt{8.586118e-01} \\
\hline
 & \texttt{4.403143e+01} & \texttt{1.863844e-02} & \texttt{2.453877e-02} & \texttt{2.462269e-02} \\
$10^{-3}$
 & \texttt{1.002975e+01} & \texttt{9.563999e-03} & \texttt{1.039723e-02} & \texttt{1.040837e-02} \\
 & \texttt{1.298019e+02} & \texttt{1.016197e-01} & \texttt{1.200189e-01} & \texttt{1.202715e-01} \\
\hline
 & \texttt{7.846784e+03} & \texttt{6.085124e-03} & \texttt{2.456530e-03} & \texttt{2.543966e-03} \\
$10^{-4}$
 & \texttt{2.571961e+03} & \texttt{2.273406e-03} & \texttt{1.049314e-03} & \texttt{1.061467e-03} \\
 & \texttt{3.324769e+04} & \texttt{2.876885e-02} & \texttt{1.232313e-02} & \texttt{1.256482e-02} \\
\hline
\end{tabular}
\caption{Ошибки для схемы \texttt{1} при $C = 1$, $\gamma = 1.4$ и~$\mu = 0.01$.}
\end{table}

\begin{table}[H]
\centering
\begin{tabular}{|c|c|c|c|c|}
\hline
\diagTH & $10^{-1}$ & $10^{-2}$ & $10^{-3}$ & $10^{-4}$ \\
\hline
 & \texttt{4.644843e+00} & \texttt{9.339581e-01} & \texttt{9.454470e-01} & \texttt{9.454472e-01} \\
$10^{-1}$
 & \texttt{1.375789e+00} & \texttt{5.171146e-01} & \texttt{5.222642e-01} & \texttt{5.222742e-01} \\
 & \texttt{1.273598e+01} & \texttt{7.654285e+00} & \texttt{6.612843e+00} & \texttt{7.274872e+00} \\
\hline
 & \texttt{3.119576e+03} & \texttt{2.020546e-01} & \texttt{2.057799e-01} & \texttt{2.058430e-01} \\
$10^{-2}$
 & \texttt{1.568071e+03} & \texttt{9.068643e-02} & \texttt{9.001549e-02} & \texttt{9.000926e-02} \\
 & \texttt{1.374665e+04} & \texttt{2.009990e+00} & \texttt{2.085705e+00} & \texttt{2.092263e+00} \\
\hline
 & \texttt{2.079895e+04} & \texttt{3.317090e-02} & \texttt{3.418815e-02} & \texttt{3.419134e-02} \\
$10^{-3}$
 & \texttt{9.170933e+03} & \texttt{1.110752e-02} & \texttt{9.974044e-03} & \texttt{9.969831e-03} \\
 & \texttt{2.839086e+04} & \texttt{3.541826e-01} & \texttt{3.880979e-01} & \texttt{3.905942e-01} \\
\hline
 & \texttt{5.783370e+04} & \texttt{6.246172e-03} & \texttt{3.812621e-03} & \texttt{3.837089e-03} \\
$10^{-4}$
 & \texttt{2.657601e+04} & \texttt{4.228745e-03} & \texttt{1.026119e-03} & \texttt{1.022018e-03} \\
 & \texttt{9.607411e+04} & \texttt{1.141844e-01} & \texttt{4.647352e-02} & \texttt{4.690111e-02} \\
\hline
\end{tabular}
\caption{Ошибки для схемы \texttt{1} при $C = 1$, $\gamma = 1$ и~$\mu = 0.001$.}
\end{table}

\begin{table}[H]
\centering
\begin{tabular}{|c|c|c|c|c|}
\hline
\diagTH & $10^{-1}$ & $10^{-2}$ & $10^{-3}$ & $10^{-4}$ \\
\hline
 & \texttt{1.364200e+01} & \texttt{4.922234e-01} & \texttt{1.393322e-01} & \texttt{1.392993e-01} \\
$10^{-1}$
 & \texttt{5.253147e+00} & \texttt{1.189438e-01} & \texttt{6.596240e-02} & \texttt{6.595421e-02} \\
 & \texttt{4.840460e+01} & \texttt{1.127427e+01} & \texttt{8.352395e-01} & \texttt{8.379238e-01} \\
\hline
 & \texttt{2.386117e+02} & \texttt{2.513277e-01} & \texttt{6.492312e-01} & \texttt{1.282036e+04} \\
$10^{-2}$
 & \texttt{6.269362e+01} & \texttt{4.784184e-02} & \texttt{9.819604e-02} & \texttt{2.072580e+03} \\
 & \texttt{5.114653e+02} & \texttt{4.572635e+00} & \texttt{4.903619e+01} & \texttt{1.001053e+07} \\
\hline
 & \texttt{3.470740e+03} & \texttt{6.347424e-03} & \texttt{5.133947e-03} & \texttt{5.121749e-03} \\
$10^{-3}$
 & \texttt{1.291329e+03} & \texttt{3.345747e-03} & \texttt{2.961004e-03} & \texttt{2.959492e-03} \\
 & \texttt{1.220914e+04} & \texttt{3.390511e-02} & \texttt{2.429845e-02} & \texttt{1.066549e-01} \\
\hline
 & \texttt{3.039044e+05} & \texttt{4.101630e-03} & \texttt{5.421482e-04} & \texttt{5.255502e-04} \\
$10^{-4}$
 & \texttt{1.185721e+05} & \texttt{2.012198e-03} & \texttt{3.035657e-04} & \texttt{3.052982e-04} \\
 & \texttt{5.708547e+05} & \texttt{2.528787e-02} & \texttt{2.512712e-03} & \texttt{2.486790e-03} \\
\hline
\end{tabular}
\caption{Ошибки для схемы \texttt{1} при $C = 10$, $\gamma = 1$ и~$\mu = 0.001$.}
\end{table}

\begin{table}[H]
\centering
\begin{tabular}{|c|c|c|c|c|}
\hline
\diagTH & $10^{-1}$ & $10^{-2}$ & $10^{-3}$ & $10^{-4}$ \\
\hline
 & \texttt{1.159302e+04} & \texttt{4.601040e-01} & \texttt{1.280137e-02} & \texttt{1.279952e-02} \\
$10^{-1}$
 & \texttt{2.919661e+03} & \texttt{1.665885e-01} & \texttt{6.609273e-03} & \texttt{6.605422e-03} \\
 & \texttt{3.974471e+04} & \texttt{1.885916e+01} & \texttt{5.661475e-02} & \texttt{5.629154e-02} \\
\hline
 & \texttt{1.754655e+03} & \texttt{1.945944e+05} & \texttt{1.128779e+08} & \texttt{1.103493e+03} \\
$10^{-2}$
 & \texttt{1.085151e+03} & \texttt{4.035348e+04} & \texttt{3.870527e+06} & \texttt{1.657320e+02} \\
 & \texttt{1.280808e+04} & \texttt{5.661580e+06} & \texttt{5.472757e+09} & \texttt{8.081585e+05} \\
\hline
 & \texttt{4.130219e+03} & \texttt{3.429071e-03} & \texttt{4.527116e-01} & \texttt{6.321251e-01} \\
$10^{-3}$
 & \texttt{2.420955e+03} & \texttt{1.138629e-03} & \texttt{1.546874e-01} & \texttt{6.010732e-02} \\
 & \texttt{1.568273e+04} & \texttt{1.894995e-02} & \texttt{6.560000e+01} & \texttt{2.288234e+02} \\
\hline
 & \texttt{1.415345e+03} & \texttt{3.654740e-03} & \texttt{3.549009e-05} & \texttt{7.319722e-02} \\
$10^{-4}$
 & \texttt{7.271865e+02} & \texttt{1.227623e-03} & \texttt{1.226319e-05} & \texttt{2.389192e-03} \\
 & \texttt{1.796642e+03} & \texttt{1.955633e-02} & \texttt{2.015581e-04} & \texttt{3.383820e+00} \\
\hline
\end{tabular}
\caption{Ошибки для схемы \texttt{1} при $C = 100$, $\gamma = 1$ и~$\mu = 0.001$.}
\end{table}

\begin{table}[H]
\centering
\begin{tabular}{|c|c|c|c|c|}
\hline
\diagTH & $10^{-1}$ & $10^{-2}$ & $10^{-3}$ & $10^{-4}$ \\
\hline
 & \texttt{1.945624e+01} & \texttt{2.062883e+00} & \texttt{1.929525e+00} & \texttt{1.916657e+00} \\
$10^{-1}$
 & \texttt{4.505354e+00} & \texttt{8.176516e-01} & \texttt{8.062316e-01} & \texttt{8.064923e-01} \\
 & \texttt{5.521580e+01} & \texttt{3.495932e+01} & \texttt{3.269724e+01} & \texttt{3.017376e+01} \\
\hline
 & \texttt{4.847890e+01} & \texttt{2.696058e-01} & \texttt{2.698981e-01} & \texttt{2.699206e-01} \\
$10^{-2}$
 & \texttt{1.105904e+01} & \texttt{1.182793e-01} & \texttt{1.191437e-01} & \texttt{1.191536e-01} \\
 & \texttt{1.476918e+02} & \texttt{1.315645e+00} & \texttt{1.329117e+00} & \texttt{1.330189e+00} \\
\hline
 & \texttt{6.645240e+02} & \texttt{3.176059e-02} & \texttt{3.437645e-02} & \texttt{3.439877e-02} \\
$10^{-3}$
 & \texttt{3.231469e+02} & \texttt{1.260999e-02} & \texttt{1.368811e-02} & \texttt{1.370164e-02} \\
 & \texttt{1.658289e+03} & \texttt{1.610896e-01} & \texttt{1.776191e-01} & \texttt{1.778211e-01} \\
\hline
 & \texttt{4.612742e+04} & \texttt{7.008618e-03} & \texttt{3.580744e-03} & \texttt{3.603998e-03} \\
$10^{-4}$
 & \texttt{2.225423e+04} & \texttt{2.602330e-03} & \texttt{1.402300e-03} & \texttt{1.415658e-03} \\
 & \texttt{3.104036e+05} & \texttt{4.368503e-02} & \texttt{1.869737e-02} & \texttt{1.888790e-02} \\
\hline
\end{tabular}
\caption{Ошибки для схемы \texttt{1} при $C = 1$, $\gamma = 1.4$ и~$\mu = 0.001$.}
\end{table}


\subsubsection{Ошибки для $V$}
\begin{table}[H]
\centering
\begin{tabular}{|c|c|c|c|c|}
\hline
\diagTH & $10^{-1}$ & $10^{-2}$ & $10^{-3}$ & $10^{-4}$ \\
\hline
 & \texttt{8.762809e-01} & \texttt{1.348669e+00} & \texttt{1.342298e+00} & \texttt{1.341862e+00} \\
$10^{-1}$
 & \texttt{5.537944e-01} & \texttt{7.229550e-01} & \texttt{7.210374e-01} & \texttt{7.210027e-01} \\
 & \texttt{4.749375e+00} & \texttt{4.483788e+00} & \texttt{4.528983e+00} & \texttt{4.531776e+00} \\
\hline
 & \texttt{4.075959e-01} & \texttt{1.772978e-01} & \texttt{1.721842e-01} & \texttt{1.721354e-01} \\
$10^{-2}$
 & \texttt{2.336403e-01} & \texttt{8.361451e-02} & \texttt{8.136064e-02} & \texttt{8.133973e-02} \\
 & \texttt{2.392059e+00} & \texttt{7.401610e-01} & \texttt{7.065944e-01} & \texttt{7.062661e-01} \\
\hline
 & \texttt{3.488753e+00} & \texttt{2.237554e-02} & \texttt{1.732869e-02} & \texttt{1.728098e-02} \\
$10^{-3}$
 & \texttt{1.405827e+00} & \texttt{1.055452e-02} & \texttt{8.324315e-03} & \texttt{8.305628e-03} \\
 & \texttt{1.105232e+01} & \texttt{1.079447e-01} & \texttt{7.341196e-02} & \texttt{7.310727e-02} \\
\hline
 & \texttt{2.986291e+00} & \texttt{7.080180e-03} & \texttt{1.771597e-03} & \texttt{1.723789e-03} \\
$10^{-4}$
 & \texttt{1.395914e+00} & \texttt{3.773419e-03} & \texttt{8.511408e-04} & \texttt{8.322779e-04} \\
 & \texttt{1.189686e+01} & \texttt{4.721378e-02} & \texttt{7.641416e-03} & \texttt{7.333983e-03} \\
\hline
\end{tabular}
\caption{Ошибки для схемы \texttt{1} при $C = 1$, $\gamma = 1$ и~$\mu = 0.1$.}
\end{table}

\begin{table}[H]
\centering
\begin{tabular}{|c|c|c|c|c|}
\hline
\diagTH & $10^{-1}$ & $10^{-2}$ & $10^{-3}$ & $10^{-4}$ \\
\hline
 & \texttt{3.121001e+00} & \texttt{1.190793e+00} & \texttt{1.189083e+00} & \texttt{1.189049e+00} \\
$10^{-1}$
 & \texttt{1.277483e+00} & \texttt{4.430741e-01} & \texttt{4.422901e-01} & \texttt{4.422825e-01} \\
 & \texttt{1.423335e+01} & \texttt{3.800451e+00} & \texttt{3.783993e+00} & \texttt{3.783820e+00} \\
\hline
 & \texttt{2.029718e+01} & \texttt{1.809768e-01} & \texttt{1.831277e-01} & \texttt{1.831519e-01} \\
$10^{-2}$
 & \texttt{1.407690e+01} & \texttt{7.032735e-02} & \texttt{7.145740e-02} & \texttt{7.146924e-02} \\
 & \texttt{7.428565e+01} & \texttt{5.231106e-01} & \texttt{5.249196e-01} & \texttt{5.249459e-01} \\
\hline
 & \texttt{1.067053e+02} & \texttt{1.720042e-02} & \texttt{2.049104e-02} & \texttt{2.052488e-02} \\
$10^{-3}$
 & \texttt{7.041326e+01} & \texttt{6.895703e-03} & \texttt{8.160589e-03} & \texttt{8.177255e-03} \\
 & \texttt{3.941923e+02} & \texttt{5.880525e-02} & \texttt{5.747657e-02} & \texttt{5.753378e-02} \\
\hline
 & \texttt{1.544312e+02} & \texttt{5.239441e-03} & \texttt{2.046592e-03} & \texttt{2.081101e-03} \\
$10^{-4}$
 & \texttt{8.564401e+01} & \texttt{2.451002e-03} & \texttt{8.139745e-04} & \texttt{8.308442e-04} \\
 & \texttt{6.108391e+02} & \texttt{2.884226e-02} & \texttt{5.768944e-03} & \texttt{5.821071e-03} \\
\hline
\end{tabular}
\caption{Ошибки для схемы \texttt{1} при $C = 10$, $\gamma = 1$ и~$\mu = 0.1$.}
\end{table}

\begin{table}[H]
\centering
\begin{tabular}{|c|c|c|c|c|}
\hline
\diagTH & $10^{-1}$ & $10^{-2}$ & $10^{-3}$ & $10^{-4}$ \\
\hline
 & \texttt{1.221450e+05} & \texttt{8.282894e-01} & \texttt{8.374431e-01} & \texttt{8.374117e-01} \\
$10^{-1}$
 & \texttt{6.355984e+04} & \texttt{3.213536e-01} & \texttt{3.274242e-01} & \texttt{3.274126e-01} \\
 & \texttt{8.721616e+05} & \texttt{3.427806e+00} & \texttt{2.522478e+00} & \texttt{2.522304e+00} \\
\hline
 & \texttt{3.582042e+02} & \texttt{2.633103e-02} & \texttt{2.586666e-02} & \texttt{2.586229e-02} \\
$10^{-2}$
 & \texttt{1.799863e+02} & \texttt{1.066156e-02} & \texttt{1.096670e-02} & \texttt{1.097184e-02} \\
 & \texttt{1.769586e+03} & \texttt{7.608705e-02} & \texttt{7.206453e-02} & \texttt{7.208729e-02} \\
\hline
 & \texttt{2.689496e+02} & \texttt{3.450924e-03} & \texttt{8.283290e-04} & \texttt{8.200688e-04} \\
$10^{-3}$
 & \texttt{1.535635e+02} & \texttt{1.847529e-03} & \texttt{5.500819e-04} & \texttt{5.615079e-04} \\
 & \texttt{2.107833e+03} & \texttt{2.579411e-02} & \texttt{3.069649e-03} & \texttt{3.221493e-03} \\
\hline
 & \texttt{2.894767e+03} & \texttt{4.201373e-03} & \texttt{6.613191e-05} & \texttt{7.799611e-05} \\
$10^{-4}$
 & \texttt{1.675389e+03} & \texttt{2.143112e-03} & \texttt{3.933929e-05} & \texttt{4.893134e-05} \\
 & \texttt{2.300055e+04} & \texttt{2.783126e-02} & \texttt{2.495662e-04} & \texttt{3.076418e-04} \\
\hline
\end{tabular}
\caption{Ошибки для схемы \texttt{1} при $C = 100$, $\gamma = 1$ и~$\mu = 0.1$.}
\end{table}

\begin{table}[H]
\centering
\begin{tabular}{|c|c|c|c|c|}
\hline
\diagTH & $10^{-1}$ & $10^{-2}$ & $10^{-3}$ & $10^{-4}$ \\
\hline
 & \texttt{1.124344e+00} & \texttt{1.111107e+00} & \texttt{1.113744e+00} & \texttt{1.114761e+00} \\
$10^{-1}$
 & \texttt{6.272306e-01} & \texttt{5.856199e-01} & \texttt{5.893024e-01} & \texttt{5.898580e-01} \\
 & \texttt{5.848721e+00} & \texttt{3.699576e+00} & \texttt{3.722526e+00} & \texttt{3.727498e+00} \\
\hline
 & \texttt{1.362551e+00} & \texttt{1.924270e-01} & \texttt{1.902728e-01} & \texttt{1.902488e-01} \\
$10^{-2}$
 & \texttt{7.256166e-01} & \texttt{8.435152e-02} & \texttt{8.411943e-02} & \texttt{8.411734e-02} \\
 & \texttt{6.238115e+00} & \texttt{6.252452e-01} & \texttt{6.131275e-01} & \texttt{6.130096e-01} \\
\hline
 & \texttt{2.231360e+00} & \texttt{2.037382e-02} & \texttt{1.865748e-02} & \texttt{1.863992e-02} \\
$10^{-3}$
 & \texttt{1.272893e+00} & \texttt{8.330838e-03} & \texttt{8.027867e-03} & \texttt{8.029704e-03} \\
 & \texttt{9.640793e+00} & \texttt{7.157309e-02} & \texttt{5.897025e-02} & \texttt{5.890070e-02} \\
\hline
 & \texttt{2.292837e+00} & \texttt{4.904446e-03} & \texttt{1.874672e-03} & \texttt{1.857532e-03} \\
$10^{-4}$
 & \texttt{1.366261e+00} & \texttt{2.897638e-03} & \texttt{7.961373e-04} & \texttt{7.978944e-04} \\
 & \texttt{1.032620e+01} & \texttt{3.084402e-02} & \texttt{5.927140e-03} & \texttt{5.856573e-03} \\
\hline
\end{tabular}
\caption{Ошибки для схемы \texttt{1} при $C = 1$, $\gamma = 1.4$ и~$\mu = 0.1$.}
\end{table}

\begin{table}[H]
\centering
\begin{tabular}{|c|c|c|c|c|}
\hline
\diagTH & $10^{-1}$ & $10^{-2}$ & $10^{-3}$ & $10^{-4}$ \\
\hline
 & \texttt{1.866822e+00} & \texttt{1.382223e+00} & \texttt{1.381771e+00} & \texttt{1.381777e+00} \\
$10^{-1}$
 & \texttt{9.253075e-01} & \texttt{5.788731e-01} & \texttt{5.785134e-01} & \texttt{5.785113e-01} \\
 & \texttt{1.105338e+01} & \texttt{6.226226e+00} & \texttt{6.457578e+00} & \texttt{6.466466e+00} \\
\hline
 & \texttt{4.516776e+00} & \texttt{2.203176e-01} & \texttt{2.244005e-01} & \texttt{2.244461e-01} \\
$10^{-2}$
 & \texttt{2.027162e+00} & \texttt{9.079278e-02} & \texttt{9.226341e-02} & \texttt{9.227849e-02} \\
 & \texttt{1.681788e+01} & \texttt{1.243512e+00} & \texttt{1.297147e+00} & \texttt{1.300685e+00} \\
\hline
 & \texttt{1.198286e+02} & \texttt{2.329110e-02} & \texttt{2.800903e-02} & \texttt{2.805475e-02} \\
$10^{-3}$
 & \texttt{8.521162e+01} & \texttt{9.087299e-03} & \texttt{1.028146e-02} & \texttt{1.029759e-02} \\
 & \texttt{4.602923e+02} & \texttt{1.606967e-01} & \texttt{1.743950e-01} & \texttt{1.757838e-01} \\
\hline
 & \texttt{3.264518e+02} & \texttt{7.135779e-03} & \texttt{2.830972e-03} & \texttt{2.876986e-03} \\
$10^{-4}$
 & \texttt{1.533002e+02} & \texttt{2.847010e-03} & \texttt{1.026755e-03} & \texttt{1.042526e-03} \\
 & \texttt{1.169740e+03} & \texttt{5.130100e-02} & \texttt{1.808264e-02} & \texttt{1.831678e-02} \\
\hline
\end{tabular}
\caption{Ошибки для схемы \texttt{1} при $C = 1$, $\gamma = 1$ и~$\mu = 0.01$.}
\end{table}

\begin{table}[H]
\centering
\begin{tabular}{|c|c|c|c|c|}
\hline
\diagTH & $10^{-1}$ & $10^{-2}$ & $10^{-3}$ & $10^{-4}$ \\
\hline
 & \texttt{3.042338e+01} & \texttt{1.492729e+00} & \texttt{1.492465e+00} & \texttt{1.492426e+00} \\
$10^{-1}$
 & \texttt{1.786137e+01} & \texttt{5.855651e-01} & \texttt{5.874105e-01} & \texttt{5.873999e-01} \\
 & \texttt{1.071383e+02} & \texttt{4.741990e+00} & \texttt{4.380465e+00} & \texttt{4.380273e+00} \\
\hline
 & \texttt{9.433527e+01} & \texttt{3.587676e-01} & \texttt{3.583525e-01} & \texttt{3.583593e-01} \\
$10^{-2}$
 & \texttt{4.305765e+01} & \texttt{1.372684e-01} & \texttt{1.376053e-01} & \texttt{1.376093e-01} \\
 & \texttt{4.490232e+02} & \texttt{1.042954e+00} & \texttt{1.038226e+00} & \texttt{1.038185e+00} \\
\hline
 & \texttt{1.898119e+02} & \texttt{3.890466e-02} & \texttt{4.065698e-02} & \texttt{4.067434e-02} \\
$10^{-3}$
 & \texttt{1.059734e+02} & \texttt{1.531493e-02} & \texttt{1.624499e-02} & \texttt{1.625832e-02} \\
 & \texttt{1.326760e+03} & \texttt{1.192319e-01} & \texttt{1.135627e-01} & \texttt{1.135633e-01} \\
\hline
 & \texttt{2.059102e+03} & \texttt{9.761243e-03} & \texttt{4.097600e-03} & \texttt{4.107048e-03} \\
$10^{-4}$
 & \texttt{1.219866e+03} & \texttt{4.072662e-03} & \texttt{1.641794e-03} & \texttt{1.654582e-03} \\
 & \texttt{1.705973e+04} & \texttt{4.882371e-02} & \texttt{1.143285e-02} & \texttt{1.139037e-02} \\
\hline
\end{tabular}
\caption{Ошибки для схемы \texttt{1} при $C = 10$, $\gamma = 1$ и~$\mu = 0.01$.}
\end{table}

\begin{table}[H]
\centering
\begin{tabular}{|c|c|c|c|c|}
\hline
\diagTH & $10^{-1}$ & $10^{-2}$ & $10^{-3}$ & $10^{-4}$ \\
\hline
 & \texttt{2.276056e+01} & \texttt{1.271856e+00} & \texttt{8.596355e-01} & \texttt{8.595963e-01} \\
$10^{-1}$
 & \texttt{1.285530e+01} & \texttt{5.541903e-01} & \texttt{3.380158e-01} & \texttt{3.380047e-01} \\
 & \texttt{1.250047e+02} & \texttt{3.173845e+01} & \texttt{2.568656e+00} & \texttt{2.568444e+00} \\
\hline
 & \texttt{3.384577e+02} & \texttt{3.772718e-01} & \texttt{2.445923e+06} & \texttt{6.896017e+04} \\
$10^{-2}$
 & \texttt{1.709174e+02} & \texttt{1.893409e-01} & \texttt{1.026981e+05} & \texttt{1.730625e+03} \\
 & \texttt{1.727007e+03} & \texttt{1.284382e+01} & \texttt{1.448051e+08} & \texttt{1.424168e+07} \\
\hline
 & \texttt{3.823883e+01} & \texttt{3.467124e-03} & \texttt{6.438681e-04} & \texttt{4.420558e-03} \\
$10^{-3}$
 & \texttt{1.859732e+01} & \texttt{1.886601e-03} & \texttt{3.545313e-04} & \texttt{4.068894e-04} \\
 & \texttt{1.696059e+02} & \texttt{2.606269e-02} & \texttt{3.333160e-03} & \texttt{5.073640e-01} \\
\hline
 & \texttt{5.404804e+02} & \texttt{7.095552e-03} & \texttt{6.103606e-05} & \texttt{9.909208e-05} \\
$10^{-4}$
 & \texttt{2.367563e+02} & \texttt{4.174000e-03} & \texttt{3.629741e-05} & \texttt{5.007827e-05} \\
 & \texttt{1.707153e+03} & \texttt{4.584491e-02} & \texttt{4.522311e-04} & \texttt{6.177592e-04} \\
\hline
\end{tabular}
\caption{Ошибки для схемы \texttt{1} при $C = 100$, $\gamma = 1$ и~$\mu = 0.01$.}
\end{table}

\begin{table}[H]
\centering
\begin{tabular}{|c|c|c|c|c|}
\hline
\diagTH & $10^{-1}$ & $10^{-2}$ & $10^{-3}$ & $10^{-4}$ \\
\hline
 & \texttt{6.755426e+00} & \texttt{2.159496e+00} & \texttt{2.171635e+00} & \texttt{2.172201e+00} \\
$10^{-1}$
 & \texttt{3.255147e+00} & \texttt{8.229624e-01} & \texttt{8.245504e-01} & \texttt{8.246957e-01} \\
 & \texttt{2.570541e+01} & \texttt{1.093887e+01} & \texttt{1.229038e+01} & \texttt{1.232680e+01} \\
\hline
 & \texttt{3.240745e+00} & \texttt{1.638207e-01} & \texttt{1.629691e-01} & \texttt{1.629594e-01} \\
$10^{-2}$
 & \texttt{1.842575e+00} & \texttt{6.974711e-02} & \texttt{7.094666e-02} & \texttt{7.095944e-02} \\
 & \texttt{1.594837e+01} & \texttt{9.778844e-01} & \texttt{9.706405e-01} & \texttt{9.709569e-01} \\
\hline
 & \texttt{2.570587e+00} & \texttt{2.109513e-02} & \texttt{2.064166e-02} & \texttt{2.063866e-02} \\
$10^{-3}$
 & \texttt{1.831652e+00} & \texttt{7.591332e-03} & \texttt{8.653362e-03} & \texttt{8.670379e-03} \\
 & \texttt{1.377001e+01} & \texttt{1.330150e-01} & \texttt{1.380703e-01} & \texttt{1.382600e-01} \\
\hline
 & \texttt{3.164836e+07} & \texttt{7.291610e-03} & \texttt{2.125377e-03} & \texttt{2.121694e-03} \\
$10^{-4}$
 & \texttt{1.159038e+07} & \texttt{3.148414e-03} & \texttt{8.733354e-04} & \texttt{8.898385e-04} \\
 & \texttt{1.326186e+08} & \texttt{4.127655e-02} & \texttt{1.432377e-02} & \texttt{1.441591e-02} \\
\hline
\end{tabular}
\caption{Ошибки для схемы \texttt{1} при $C = 1$, $\gamma = 1.4$ и~$\mu = 0.01$.}
\end{table}

\begin{table}[H]
\centering
\begin{tabular}{|c|c|c|c|c|}
\hline
\diagTH & $10^{-1}$ & $10^{-2}$ & $10^{-3}$ & $10^{-4}$ \\
\hline
 & \texttt{2.499247e+00} & \texttt{1.625805e+00} & \texttt{1.636371e+00} & \texttt{1.636625e+00} \\
$10^{-1}$
 & \texttt{1.110041e+00} & \texttt{6.276437e-01} & \texttt{6.299163e-01} & \texttt{6.300246e-01} \\
 & \texttt{1.324640e+01} & \texttt{7.182896e+00} & \texttt{8.039431e+00} & \texttt{8.059145e+00} \\
\hline
 & \texttt{6.643208e+01} & \texttt{2.979725e-01} & \texttt{2.948896e-01} & \texttt{2.948276e-01} \\
$10^{-2}$
 & \texttt{4.215700e+01} & \texttt{1.168625e-01} & \texttt{1.183004e-01} & \texttt{1.183150e-01} \\
 & \texttt{2.874351e+02} & \texttt{1.909283e+00} & \texttt{1.949970e+00} & \texttt{1.954959e+00} \\
\hline
 & \texttt{1.527744e+02} & \texttt{3.531857e-02} & \texttt{4.131314e-02} & \texttt{4.137837e-02} \\
$10^{-3}$
 & \texttt{1.174078e+02} & \texttt{1.170380e-02} & \texttt{1.301039e-02} & \texttt{1.302667e-02} \\
 & \texttt{6.337207e+02} & \texttt{3.771821e-01} & \texttt{3.847270e-01} & \texttt{3.852552e-01} \\
\hline
 & \texttt{4.051511e+02} & \texttt{9.020232e-03} & \texttt{4.510097e-03} & \texttt{4.588108e-03} \\
$10^{-4}$
 & \texttt{1.976979e+02} & \texttt{2.902244e-03} & \texttt{1.310970e-03} & \texttt{1.327108e-03} \\
 & \texttt{1.466415e+03} & \texttt{1.248130e-01} & \texttt{4.592336e-02} & \texttt{4.648523e-02} \\
\hline
\end{tabular}
\caption{Ошибки для схемы \texttt{1} при $C = 1$, $\gamma = 1$ и~$\mu = 0.001$.}
\end{table}

\begin{table}[H]
\centering
\begin{tabular}{|c|c|c|c|c|}
\hline
\diagTH & $10^{-1}$ & $10^{-2}$ & $10^{-3}$ & $10^{-4}$ \\
\hline
 & \texttt{2.187930e+01} & \texttt{1.507703e+00} & \texttt{1.532674e+00} & \texttt{1.532652e+00} \\
$10^{-1}$
 & \texttt{1.196209e+01} & \texttt{6.164964e-01} & \texttt{6.078360e-01} & \texttt{6.078247e-01} \\
 & \texttt{1.139058e+02} & \texttt{2.585269e+01} & \texttt{4.467050e+00} & \texttt{4.466852e+00} \\
\hline
 & \texttt{4.796593e+01} & \texttt{4.283601e-01} & \texttt{3.423500e-01} & \texttt{6.059856e+04} \\
$10^{-2}$
 & \texttt{2.561751e+01} & \texttt{1.600718e-01} & \texttt{1.337274e-01} & \texttt{3.490504e+04} \\
 & \texttt{1.597774e+02} & \texttt{8.536486e+00} & \texttt{1.184800e+01} & \texttt{1.912717e+05} \\
\hline
 & \texttt{2.644110e+02} & \texttt{4.346291e-02} & \texttt{4.449849e-02} & \texttt{4.450749e-02} \\
$10^{-3}$
 & \texttt{1.397856e+02} & \texttt{1.710117e-02} & \texttt{1.781895e-02} & \texttt{1.783044e-02} \\
 & \texttt{1.844355e+03} & \texttt{1.325574e-01} & \texttt{1.240966e-01} & \texttt{1.263534e-01} \\
\hline
 & \texttt{3.235962e+03} & \texttt{1.076652e-02} & \texttt{4.483426e-03} & \texttt{4.477513e-03} \\
$10^{-4}$
 & \texttt{1.556500e+03} & \texttt{4.788524e-03} & \texttt{1.802351e-03} & \texttt{1.812800e-03} \\
 & \texttt{2.108758e+04} & \texttt{5.702832e-02} & \texttt{1.247649e-02} & \texttt{1.237491e-02} \\
\hline
\end{tabular}
\caption{Ошибки для схемы \texttt{1} при $C = 10$, $\gamma = 1$ и~$\mu = 0.001$.}
\end{table}

\begin{table}[H]
\centering
\begin{tabular}{|c|c|c|c|c|}
\hline
\diagTH & $10^{-1}$ & $10^{-2}$ & $10^{-3}$ & $10^{-4}$ \\
\hline
 & \texttt{1.146038e+02} & \texttt{2.681551e+00} & \texttt{8.618695e-01} & \texttt{8.618280e-01} \\
$10^{-1}$
 & \texttt{7.974051e+01} & \texttt{1.145585e+00} & \texttt{3.390848e-01} & \texttt{3.390752e-01} \\
 & \texttt{3.096951e+02} & \texttt{1.159571e+02} & \texttt{2.573314e+00} & \texttt{2.573109e+00} \\
\hline
 & \texttt{3.387665e+02} & \texttt{4.276125e+04} & \texttt{2.081394e+06} & \texttt{5.476387e+02} \\
$10^{-2}$
 & \texttt{1.811969e+02} & \texttt{2.184386e+04} & \texttt{1.199855e+06} & \texttt{1.739579e+02} \\
 & \texttt{1.710440e+03} & \texttt{3.253859e+05} & \texttt{6.288303e+06} & \texttt{6.593405e+04} \\
\hline
 & \texttt{1.228847e+03} & \texttt{3.477593e-03} & \texttt{3.195484e-01} & \texttt{3.989721e-02} \\
$10^{-3}$
 & \texttt{7.998841e+02} & \texttt{1.899974e-03} & \texttt{1.530316e-01} & \texttt{1.755728e-02} \\
 & \texttt{1.019607e+04} & \texttt{2.616471e-02} & \texttt{9.128367e+01} & \texttt{3.842754e+01} \\
\hline
 & \texttt{6.044344e+02} & \texttt{8.707148e-03} & \texttt{7.171132e-05} & \texttt{7.957116e-02} \\
$10^{-4}$
 & \texttt{3.135619e+02} & \texttt{4.966010e-03} & \texttt{4.457280e-05} & \texttt{1.110378e-03} \\
 & \texttt{2.078092e+03} & \texttt{5.306976e-02} & \texttt{5.306925e-04} & \texttt{9.694426e+00} \\
\hline
\end{tabular}
\caption{Ошибки для схемы \texttt{1} при $C = 100$, $\gamma = 1$ и~$\mu = 0.001$.}
\end{table}

\begin{table}[H]
\centering
\begin{tabular}{|c|c|c|c|c|}
\hline
\diagTH & $10^{-1}$ & $10^{-2}$ & $10^{-3}$ & $10^{-4}$ \\
\hline
 & \texttt{1.069411e+01} & \texttt{2.846656e+00} & \texttt{2.851724e+00} & \texttt{2.852341e+00} \\
$10^{-1}$
 & \texttt{3.903570e+00} & \texttt{1.082564e+00} & \texttt{1.097177e+00} & \texttt{1.097656e+00} \\
 & \texttt{4.161031e+01} & \texttt{3.041534e+01} & \texttt{3.482685e+01} & \texttt{3.622985e+01} \\
\hline
 & \texttt{4.075133e+00} & \texttt{3.006917e-01} & \texttt{3.046907e-01} & \texttt{3.047425e-01} \\
$10^{-2}$
 & \texttt{2.039748e+00} & \texttt{1.162158e-01} & \texttt{1.183053e-01} & \texttt{1.183294e-01} \\
 & \texttt{1.997139e+01} & \texttt{1.598208e+00} & \texttt{1.590334e+00} & \texttt{1.590432e+00} \\
\hline
 & \texttt{1.144955e+05} & \texttt{3.294221e-02} & \texttt{3.875467e-02} & \texttt{3.881834e-02} \\
$10^{-3}$
 & \texttt{4.583945e+04} & \texttt{1.021947e-02} & \texttt{1.214773e-02} & \texttt{1.217296e-02} \\
 & \texttt{4.267857e+05} & \texttt{2.115227e-01} & \texttt{2.116635e-01} & \texttt{2.117360e-01} \\
\hline
 & \texttt{9.867371e+09} & \texttt{6.760864e-03} & \texttt{3.996561e-03} & \texttt{4.066082e-03} \\
$10^{-4}$
 & \texttt{3.120673e+09} & \texttt{3.192129e-03} & \texttt{1.210434e-03} & \texttt{1.235945e-03} \\
 & \texttt{4.400807e+10} & \texttt{5.499431e-02} & \texttt{2.210086e-02} & \texttt{2.218519e-02} \\
\hline
\end{tabular}
\caption{Ошибки для схемы \texttt{1} при $C = 1$, $\gamma = 1.4$ и~$\mu = 0.001$.}
\end{table}


\subsubsection{Ошибки для $G$ методом вложенных сеток}
\begin{table}[H]
\centering
\begin{tabular}{|c|c|c|c|c|}
\hline
\diagTHk & $1/2$ & $1/4$ & $1/8$ & $u$ \\
\hline
 & \texttt{1.253369e+01} & \texttt{1.207643e+01} & \texttt{1.194145e+01} & \texttt{1.181652e+01} \\
$10^{-1}$
 & \texttt{5.340344e+00} & \texttt{5.100815e+00} & \texttt{4.981200e+00} & \texttt{4.893319e+00} \\
 & \texttt{4.067799e+01} & \texttt{4.025670e+01} & \texttt{4.003381e+01} & \texttt{3.968220e+01} \\
\hline
 & \texttt{8.898848e-02} & \texttt{1.329437e-01} & \texttt{1.548267e-01} & \texttt{1.766651e-01} \\
$10^{-2}$
 & \texttt{5.153601e-02} & \texttt{7.543472e-02} & \texttt{8.692465e-02} & \texttt{9.811932e-02} \\
 & \texttt{4.028767e-01} & \texttt{5.979688e-01} & \texttt{6.933094e-01} & \texttt{7.879295e-01} \\
\hline
 & \texttt{8.783186e-03} & \texttt{1.317308e-02} & \texttt{1.536756e-02} & \texttt{1.756171e-02} \\
$10^{-3}$
 & \texttt{4.522259e-03} & \texttt{6.767357e-03} & \texttt{7.885920e-03} & \texttt{9.001833e-03} \\
 & \texttt{4.090053e-02} & \texttt{6.138242e-02} & \texttt{7.156724e-02} & \texttt{8.186950e-02} \\
\hline
\end{tabular}
\caption{Ошибки для схемы 1 при $C = 1$, $\gamma = 1$ и~$\mu = 0.1$.}
\end{table}

\begin{table}[H]
\centering
\begin{tabular}{|c|c|c|c|c|}
\hline
\diagTHk & $1/2$ & $1/4$ & $1/8$ & $u$ \\
\hline
 & \texttt{3.166500e+00} & \texttt{3.177658e+00} & \texttt{3.196248e+00} & \texttt{3.222515e+00} \\
$10^{-1}$
 & \texttt{1.086030e+00} & \texttt{1.100972e+00} & \texttt{1.109850e+00} & \texttt{1.120161e+00} \\
 & \texttt{1.353279e+01} & \texttt{1.341818e+01} & \texttt{1.338757e+01} & \texttt{1.339440e+01} \\
\hline
 & \texttt{1.061715e-02} & \texttt{1.654810e-02} & \texttt{1.968695e-02} & \texttt{2.294542e-02} \\
$10^{-2}$
 & \texttt{5.466373e-03} & \texttt{8.410225e-03} & \texttt{9.951641e-03} & \texttt{1.154502e-02} \\
 & \texttt{5.707844e-02} & \texttt{8.097355e-02} & \texttt{9.202206e-02} & \texttt{1.030876e-01} \\
\hline
 & \texttt{1.298833e-03} & \texttt{1.955559e-03} & \texttt{2.285755e-03} & \texttt{2.617173e-03} \\
$10^{-3}$
 & \texttt{6.434264e-04} & \texttt{9.686715e-04} & \texttt{1.132189e-03} & \texttt{1.296306e-03} \\
 & \texttt{5.213734e-03} & \texttt{7.849615e-03} & \texttt{9.136063e-03} & \texttt{1.041279e-02} \\
\hline
\end{tabular}
\caption{Ошибки для схемы 1 при $C = 10$, $\gamma = 1$ и~$\mu = 0.1$.}
\end{table}

\begin{table}[H]
\centering
\begin{tabular}{|c|c|c|c|c|}
\hline
\diagTHk & $1/2$ & $1/4$ & $1/8$ & $u$ \\
\hline
 & \texttt{4.851437e+04} & \texttt{4.851431e+04} & \texttt{4.851431e+04} & \texttt{4.851430e+04} \\
$10^{-1}$
 & \texttt{2.208043e+04} & \texttt{2.208050e+04} & \texttt{2.208050e+04} & \texttt{2.208050e+04} \\
 & \texttt{3.103523e+05} & \texttt{3.103531e+05} & \texttt{3.103531e+05} & \texttt{3.103531e+05} \\
\hline
 & \texttt{2.900661e-03} & \texttt{3.727833e-03} & \texttt{3.991865e-03} & \texttt{4.145018e-03} \\
$10^{-2}$
 & \texttt{1.019272e-03} & \texttt{1.363252e-03} & \texttt{1.500089e-03} & \texttt{1.611664e-03} \\
 & \texttt{1.505803e-02} & \texttt{1.843374e-02} & \texttt{1.959859e-02} & \texttt{2.015021e-02} \\
\hline
 & \texttt{1.177425e-04} & \texttt{1.748300e-04} & \texttt{2.027989e-04} & \texttt{2.303365e-04} \\
$10^{-3}$
 & \texttt{6.141827e-05} & \texttt{9.128915e-05} & \texttt{1.059834e-04} & \texttt{1.205005e-04} \\
 & \texttt{2.632870e-04} & \texttt{3.706553e-04} & \texttt{4.189140e-04} & \texttt{4.640649e-04} \\
\hline
\end{tabular}
\caption{Ошибки для схемы 1 при $C = 100$, $\gamma = 1$ и~$\mu = 0.1$.}
\end{table}

\begin{table}[H]
\centering
\begin{tabular}{|c|c|c|c|c|}
\hline
\diagTHk & $1/2$ & $1/4$ & $1/8$ & $u$ \\
\hline
 & \texttt{9.601373e+00} & \texttt{9.281939e+00} & \texttt{9.180867e+00} & \texttt{9.118165e+00} \\
$10^{-1}$
 & \texttt{3.617754e+00} & \texttt{3.451460e+00} & \texttt{3.390602e+00} & \texttt{3.344701e+00} \\
 & \texttt{3.497895e+01} & \texttt{3.436551e+01} & \texttt{3.411839e+01} & \texttt{3.390989e+01} \\
\hline
 & \texttt{4.610781e-02} & \texttt{6.845632e-02} & \texttt{7.947222e-02} & \texttt{9.039186e-02} \\
$10^{-2}$
 & \texttt{2.826461e-02} & \texttt{4.121450e-02} & \texttt{4.740317e-02} & \texttt{5.340469e-02} \\
 & \texttt{2.257737e-01} & \texttt{3.305196e-01} & \texttt{3.862826e-01} & \texttt{4.388565e-01} \\
\hline
 & \texttt{4.387671e-03} & \texttt{6.577270e-03} & \texttt{7.671046e-03} & \texttt{8.764154e-03} \\
$10^{-3}$
 & \texttt{2.432897e-03} & \texttt{3.639236e-03} & \texttt{4.239884e-03} & \texttt{4.838855e-03} \\
 & \texttt{2.228184e-02} & \texttt{3.340479e-02} & \texttt{3.892185e-02} & \texttt{4.443798e-02} \\
\hline
\end{tabular}
\caption{Ошибки для схемы 1 при $C = 1$, $\gamma = 1.4$ и~$\mu = 0.1$.}
\end{table}

\begin{table}[H]
\centering
\begin{tabular}{|c|c|c|c|c|}
\hline
\diagTHk & $1/2$ & $1/4$ & $1/8$ & $u$ \\
\hline
 & \texttt{4.386061e+00} & \texttt{4.470084e+00} & \texttt{4.516889e+00} & \texttt{4.564166e+00} \\
$10^{-1}$
 & \texttt{1.537111e+00} & \texttt{1.557131e+00} & \texttt{1.565479e+00} & \texttt{1.573963e+00} \\
 & \texttt{1.721237e+01} & \texttt{1.673421e+01} & \texttt{1.619988e+01} & \texttt{1.560271e+01} \\
\hline
 & \texttt{8.233537e-02} & \texttt{1.265394e-01} & \texttt{1.522039e-01} & \texttt{1.792192e-01} \\
$10^{-2}$
 & \texttt{4.078537e-02} & \texttt{6.131279e-02} & \texttt{7.163398e-02} & \texttt{8.201671e-02} \\
 & \texttt{6.790384e-01} & \texttt{1.017965e+00} & \texttt{1.202701e+00} & \texttt{1.382530e+00} \\
\hline
 & \texttt{1.109506e-02} & \texttt{1.676764e-02} & \texttt{1.963546e-02} & \texttt{2.252434e-02} \\
$10^{-3}$
 & \texttt{4.307235e-03} & \texttt{6.472682e-03} & \texttt{7.558475e-03} & \texttt{8.646357e-03} \\
 & \texttt{8.997875e-02} & \texttt{1.362690e-01} & \texttt{1.599747e-01} & \texttt{1.838053e-01} \\
\hline
\end{tabular}
\caption{Ошибки для схемы 1 при $C = 1$, $\gamma = 1$ и~$\mu = 0.01$.}
\end{table}

\begin{table}[H]
\centering
\begin{tabular}{|c|c|c|c|c|}
\hline
\diagTHk & $1/2$ & $1/4$ & $1/8$ & $u$ \\
\hline
 & \texttt{2.979399e+01} & \texttt{2.953241e+01} & \texttt{2.951620e+01} & \texttt{2.949404e+01} \\
$10^{-1}$
 & \texttt{1.367704e+01} & \texttt{1.364175e+01} & \texttt{1.364367e+01} & \texttt{1.364401e+01} \\
 & \texttt{7.151111e+01} & \texttt{7.087184e+01} & \texttt{7.077172e+01} & \texttt{7.068992e+01} \\
\hline
 & \texttt{1.872358e-02} & \texttt{2.742555e-02} & \texttt{3.297214e-02} & \texttt{3.883304e-02} \\
$10^{-2}$
 & \texttt{9.993156e-03} & \texttt{1.568339e-02} & \texttt{1.874661e-02} & \texttt{2.198733e-02} \\
 & \texttt{1.110678e-01} & \texttt{1.598628e-01} & \texttt{1.832515e-01} & \texttt{2.085437e-01} \\
\hline
 & \texttt{2.333335e-03} & \texttt{3.520898e-03} & \texttt{4.120164e-03} & \texttt{4.723141e-03} \\
$10^{-3}$
 & \texttt{1.311524e-03} & \texttt{1.980376e-03} & \texttt{2.318370e-03} & \texttt{2.658904e-03} \\
 & \texttt{1.077181e-02} & \texttt{1.615044e-02} & \texttt{1.887700e-02} & \texttt{2.164434e-02} \\
\hline
\end{tabular}
\caption{Ошибки для схемы 1 при $C = 10$, $\gamma = 1$ и~$\mu = 0.01$.}
\end{table}

\begin{table}[H]
\centering
\begin{tabular}{|c|c|c|c|c|}
\hline
\diagTHk & $1/2$ & $1/4$ & $1/8$ & $u$ \\
\hline
 & \texttt{2.333123e+16} & \texttt{8.797645e+03} & \texttt{8.142750e+04} & \texttt{2.464089e+03} \\
$10^{-1}$
 & \texttt{2.331347e+16} & \texttt{3.338920e+03} & \texttt{5.332465e+04} & \texttt{9.520705e+02} \\
 & \texttt{2.331365e+16} & \texttt{2.241914e+04} & \texttt{1.006713e+05} & \texttt{1.004560e+04} \\
\hline
 & \texttt{4.774135e-01} & \texttt{4.783216e-01} & \texttt{3.889429e-01} & \texttt{3.790905e-01} \\
$10^{-2}$
 & \texttt{2.611464e-01} & \texttt{2.397932e-01} & \texttt{2.371807e-01} & \texttt{2.368079e-01} \\
 & \texttt{2.182054e+01} & \texttt{6.787065e+00} & \texttt{3.375505e+00} & \texttt{3.709077e+00} \\
\hline
 & \texttt{9.727123e-05} & \texttt{1.346982e-04} & \texttt{1.499038e-04} & \texttt{1.629036e-04} \\
$10^{-3}$
 & \texttt{4.985512e-05} & \texttt{6.919995e-05} & \texttt{7.720359e-05} & \texttt{8.384069e-05} \\
 & \texttt{2.185211e-04} & \texttt{2.843040e-04} & \texttt{3.058941e-04} & \texttt{3.215977e-04} \\
\hline
\end{tabular}
\caption{Ошибки для схемы 1 при $C = 100$, $\gamma = 1$ и~$\mu = 0.01$.}
\end{table}

\begin{table}[H]
\centering
\begin{tabular}{|c|c|c|c|c|}
\hline
\diagTHk & $1/2$ & $1/4$ & $1/8$ & $u$ \\
\hline
 & \texttt{1.109840e+01} & \texttt{1.085101e+01} & \texttt{1.085446e+01} & \texttt{1.081077e+01} \\
$10^{-1}$
 & \texttt{3.021398e+00} & \texttt{2.923169e+00} & \texttt{2.935298e+00} & \texttt{2.953611e+00} \\
 & \texttt{2.937721e+01} & \texttt{2.883205e+01} & \texttt{2.856370e+01} & \texttt{2.850794e+01} \\
\hline
 & \texttt{8.159091e-02} & \texttt{1.306283e-01} & \texttt{1.582813e-01} & \texttt{1.886602e-01} \\
$10^{-2}$
 & \texttt{4.329276e-02} & \texttt{6.639171e-02} & \texttt{7.854616e-02} & \texttt{9.125062e-02} \\
 & \texttt{3.724326e-01} & \texttt{5.892927e-01} & \texttt{7.098027e-01} & \texttt{8.469457e-01} \\
\hline
 & \texttt{1.202429e-02} & \texttt{1.821785e-02} & \texttt{2.136209e-02} & \texttt{2.453877e-02} \\
$10^{-3}$
 & \texttt{5.140695e-03} & \texttt{7.753792e-03} & \texttt{9.071627e-03} & \texttt{1.039723e-02} \\
 & \texttt{5.853451e-02} & \texttt{8.889199e-02} & \texttt{1.043583e-01} & \texttt{1.200189e-01} \\
\hline
\end{tabular}
\caption{Ошибки для схемы 1 при $C = 1$, $\gamma = 1.4$ и~$\mu = 0.01$.}
\end{table}

\begin{table}[H]
\centering
\begin{tabular}{|c|c|c|c|c|}
\hline
\diagTHk & $1/2$ & $1/4$ & $1/8$ & $u$ \\
\hline
 & \texttt{4.406416e+00} & \texttt{4.529485e+00} & \texttt{4.587279e+00} & \texttt{4.644843e+00} \\
$10^{-1}$
 & \texttt{1.304921e+00} & \texttt{1.322575e+00} & \texttt{1.354848e+00} & \texttt{1.375789e+00} \\
 & \texttt{1.153001e+01} & \texttt{1.253100e+01} & \texttt{1.287264e+01} & \texttt{1.273598e+01} \\
\hline
 & \texttt{1.492147e-01} & \texttt{1.861235e-01} & \texttt{1.949450e-01} & \texttt{2.020546e-01} \\
$10^{-2}$
 & \texttt{4.699934e-02} & \texttt{6.888286e-02} & \texttt{7.972697e-02} & \texttt{9.068643e-02} \\
 & \texttt{1.384848e+00} & \texttt{1.739787e+00} & \texttt{1.885459e+00} & \texttt{2.009990e+00} \\
\hline
 & \texttt{1.631037e-02} & \texttt{2.504733e-02} & \texttt{2.956058e-02} & \texttt{3.418815e-02} \\
$10^{-3}$
 & \texttt{4.940809e-03} & \texttt{7.445129e-03} & \texttt{8.706417e-03} & \texttt{9.974044e-03} \\
 & \texttt{1.866439e-01} & \texttt{2.850847e-01} & \texttt{3.359444e-01} & \texttt{3.880979e-01} \\
\hline
\end{tabular}
\caption{Ошибки для схемы 1 при $C = 1$, $\gamma = 1$ и~$\mu = 0.001$.}
\end{table}

\begin{table}[H]
\centering
\begin{tabular}{|c|c|c|c|c|}
\hline
\diagTHk & $1/2$ & $1/4$ & $1/8$ & $u$ \\
\hline
 & \texttt{1.334468e+01} & \texttt{1.338180e+01} & \texttt{1.329570e+01} & \texttt{1.364200e+01} \\
$10^{-1}$
 & \texttt{5.193881e+00} & \texttt{5.185703e+00} & \texttt{5.198800e+00} & \texttt{5.253147e+00} \\
 & \texttt{4.799195e+01} & \texttt{4.830389e+01} & \texttt{4.809707e+01} & \texttt{4.840460e+01} \\
\hline
 & \texttt{2.101919e-01} & \texttt{2.446505e-01} & \texttt{2.477356e-01} & \texttt{2.513277e-01} \\
$10^{-2}$
 & \texttt{4.056075e-02} & \texttt{4.408270e-02} & \texttt{4.580417e-02} & \texttt{4.784184e-02} \\
 & \texttt{3.849113e+00} & \texttt{4.229614e+00} & \texttt{4.388390e+00} & \texttt{4.572635e+00} \\
\hline
 & \texttt{2.533496e-03} & \texttt{3.825285e-03} & \texttt{4.477640e-03} & \texttt{5.133947e-03} \\
$10^{-3}$
 & \texttt{1.457269e-03} & \texttt{2.202453e-03} & \texttt{2.579888e-03} & \texttt{2.961004e-03} \\
 & \texttt{1.204363e-02} & \texttt{1.808909e-02} & \texttt{2.116451e-02} & \texttt{2.429845e-02} \\
\hline
\end{tabular}
\caption{Ошибки для схемы 1 при $C = 10$, $\gamma = 1$ и~$\mu = 0.001$.}
\end{table}

\begin{table}[H]
\centering
\begin{tabular}{|c|c|c|c|c|}
\hline
\diagTHk & $1/2$ & $1/4$ & $1/8$ & $u$ \\
\hline
 & \texttt{2.802788e+04} & \texttt{1.009402e+04} & \texttt{3.876481e+05} & \texttt{1.159302e+04} \\
$10^{-1}$
 & \texttt{9.920653e+03} & \texttt{4.099451e+03} & \texttt{1.357320e+05} & \texttt{2.919661e+03} \\
 & \texttt{4.565943e+04} & \texttt{4.167557e+04} & \texttt{1.204900e+06} & \texttt{3.974471e+04} \\
\hline
 & \texttt{6.723781e+05} & \texttt{2.299409e+05} & \texttt{1.945946e+05} & \texttt{1.945944e+05} \\
$10^{-2}$
 & \texttt{1.000930e+05} & \texttt{4.847551e+04} & \texttt{4.035354e+04} & \texttt{4.035348e+04} \\
 & \texttt{6.819297e+06} & \texttt{5.087345e+06} & \texttt{5.661598e+06} & \texttt{5.661580e+06} \\
\hline
 & \texttt{7.227391e-01} & \texttt{5.462907e-01} & \texttt{4.293167e-01} & \texttt{4.527116e-01} \\
$10^{-3}$
 & \texttt{1.677347e-01} & \texttt{1.542000e-01} & \texttt{1.544521e-01} & \texttt{1.546874e-01} \\
 & \texttt{1.453544e+02} & \texttt{6.048373e+01} & \texttt{6.362970e+01} & \texttt{6.560000e+01} \\
\hline
\end{tabular}
\caption{Ошибки для схемы 1 при $C = 100$, $\gamma = 1$ и~$\mu = 0.001$.}
\end{table}

\begin{table}[H]
\centering
\begin{tabular}{|c|c|c|c|c|}
\hline
\diagTHk & $1/2$ & $1/4$ & $1/8$ & $u$ \\
\hline
 & \texttt{2.156395e+01} & \texttt{1.981044e+01} & \texttt{1.971110e+01} & \texttt{1.945624e+01} \\
$10^{-1}$
 & \texttt{5.129213e+00} & \texttt{4.593219e+00} & \texttt{4.569608e+00} & \texttt{4.505354e+00} \\
 & \texttt{5.876904e+01} & \texttt{5.585198e+01} & \texttt{5.551285e+01} & \texttt{5.521580e+01} \\
\hline
 & \texttt{1.232351e-01} & \texttt{1.901606e-01} & \texttt{2.281318e-01} & \texttt{2.696058e-01} \\
$10^{-2}$
 & \texttt{5.828806e-02} & \texttt{8.683748e-02} & \texttt{1.020365e-01} & \texttt{1.182793e-01} \\
 & \texttt{8.069071e-01} & \texttt{1.011055e+00} & \texttt{1.151504e+00} & \texttt{1.315645e+00} \\
\hline
 & \texttt{1.675787e-02} & \texttt{2.545043e-02} & \texttt{2.988419e-02} & \texttt{3.437645e-02} \\
$10^{-3}$
 & \texttt{6.724408e-03} & \texttt{1.017364e-02} & \texttt{1.192235e-02} & \texttt{1.368811e-02} \\
 & \texttt{8.615130e-02} & \texttt{1.311386e-01} & \texttt{1.541812e-01} & \texttt{1.776191e-01} \\
\hline
\end{tabular}
\caption{Ошибки для схемы 1 при $C = 1$, $\gamma = 1.4$ и~$\mu = 0.001$.}
\end{table}


\subsubsection{Ошибки для $V$ методом вложенных сеток}
\begin{table}[H]
\centering
\begin{tabular}{|c|c|c|c|c|}
\hline
\diagTHk & $1/2$ & $1/4$ & $1/8$ & $u$ \\
\hline
 & \texttt{1.497440e+00} & \texttt{1.117859e+00} & \texttt{9.757665e-01} & \texttt{8.762809e-01} \\
$10^{-1}$
 & \texttt{9.202854e-01} & \texttt{7.198558e-01} & \texttt{6.299974e-01} & \texttt{5.537944e-01} \\
 & \texttt{7.326097e+00} & \texttt{5.698887e+00} & \texttt{5.109705e+00} & \texttt{4.749375e+00} \\
\hline
 & \texttt{9.058192e-02} & \texttt{1.350214e-01} & \texttt{1.565155e-01} & \texttt{1.772978e-01} \\
$10^{-2}$
 & \texttt{4.198269e-02} & \texttt{6.283105e-02} & \texttt{7.322907e-02} & \texttt{8.361451e-02} \\
 & \texttt{3.795082e-01} & \texttt{5.624288e-01} & \texttt{6.519466e-01} & \texttt{7.401610e-01} \\
\hline
 & \texttt{8.691211e-03} & \texttt{1.301658e-02} & \texttt{1.517456e-02} & \texttt{1.732869e-02} \\
$10^{-3}$
 & \texttt{4.162623e-03} & \texttt{6.243603e-03} & \texttt{7.283994e-03} & \texttt{8.324315e-03} \\
 & \texttt{3.673740e-02} & \texttt{5.508296e-02} & \texttt{6.424958e-02} & \texttt{7.341196e-02} \\
\hline
\end{tabular}
\caption{Ошибки для схемы 1 при $C = 1$, $\gamma = 1$ и~$\mu = 0.1$.}
\end{table}

\begin{table}[H]
\centering
\begin{tabular}{|c|c|c|c|c|}
\hline
\diagTHk & $1/2$ & $1/4$ & $1/8$ & $u$ \\
\hline
 & \texttt{3.883598e+00} & \texttt{3.492513e+00} & \texttt{3.317919e+00} & \texttt{3.121001e+00} \\
$10^{-1}$
 & \texttt{1.506742e+00} & \texttt{1.399536e+00} & \texttt{1.345208e+00} & \texttt{1.277483e+00} \\
 & \texttt{1.723900e+01} & \texttt{1.596138e+01} & \texttt{1.521452e+01} & \texttt{1.423335e+01} \\
\hline
 & \texttt{8.476121e-02} & \texttt{1.310730e-01} & \texttt{1.555208e-01} & \texttt{1.809768e-01} \\
$10^{-2}$
 & \texttt{3.241451e-02} & \texttt{5.052189e-02} & \texttt{6.018818e-02} & \texttt{7.032735e-02} \\
 & \texttt{2.501765e-01} & \texttt{3.827546e-01} & \texttt{4.518025e-01} & \texttt{5.231106e-01} \\
\hline
 & \texttt{1.015797e-02} & \texttt{1.530209e-02} & \texttt{1.789090e-02} & \texttt{2.049104e-02} \\
$10^{-3}$
 & \texttt{4.039726e-03} & \texttt{6.089790e-03} & \texttt{7.122569e-03} & \texttt{8.160589e-03} \\
 & \texttt{2.853802e-02} & \texttt{4.295578e-02} & \texttt{5.020311e-02} & \texttt{5.747657e-02} \\
\hline
\end{tabular}
\caption{Ошибки для схемы 1 при $C = 10$, $\gamma = 1$ и~$\mu = 0.1$.}
\end{table}

\begin{table}[H]
\centering
\begin{tabular}{|c|c|c|c|c|}
\hline
\diagTHk & $1/2$ & $1/4$ & $1/8$ & $u$ \\
\hline
 & \texttt{1.221467e+05} & \texttt{1.221454e+05} & \texttt{1.221450e+05} & \texttt{1.221450e+05} \\
$10^{-1}$
 & \texttt{6.356061e+04} & \texttt{6.355995e+04} & \texttt{6.355983e+04} & \texttt{6.355984e+04} \\
 & \texttt{8.721820e+05} & \texttt{8.721642e+05} & \texttt{8.721615e+05} & \texttt{8.721616e+05} \\
\hline
 & \texttt{1.771055e-02} & \texttt{2.332227e-02} & \texttt{2.522033e-02} & \texttt{2.633103e-02} \\
$10^{-2}$
 & \texttt{6.872338e-03} & \texttt{9.199342e-03} & \texttt{1.006184e-02} & \texttt{1.066156e-02} \\
 & \texttt{5.473151e-02} & \texttt{6.987523e-02} & \texttt{7.425956e-02} & \texttt{7.608705e-02} \\
\hline
 & \texttt{4.952793e-04} & \texttt{6.853840e-04} & \texttt{7.633556e-04} & \texttt{8.283290e-04} \\
$10^{-3}$
 & \texttt{2.943690e-04} & \texttt{4.274528e-04} & \texttt{4.901146e-04} & \texttt{5.500819e-04} \\
 & \texttt{1.605632e-03} & \texttt{2.354480e-03} & \texttt{2.714236e-03} & \texttt{3.069649e-03} \\
\hline
\end{tabular}
\caption{Ошибки для схемы 1 при $C = 100$, $\gamma = 1$ и~$\mu = 0.1$.}
\end{table}

\begin{table}[H]
\centering
\begin{tabular}{|c|c|c|c|c|}
\hline
\diagTHk & $1/2$ & $1/4$ & $1/8$ & $u$ \\
\hline
 & \texttt{1.426489e+00} & \texttt{1.085351e+00} & \texttt{1.098266e+00} & \texttt{1.124344e+00} \\
$10^{-1}$
 & \texttt{8.835346e-01} & \texttt{7.284267e-01} & \texttt{6.657531e-01} & \texttt{6.272306e-01} \\
 & \texttt{8.162799e+00} & \texttt{6.519334e+00} & \texttt{5.979254e+00} & \texttt{5.848721e+00} \\
\hline
 & \texttt{9.757992e-02} & \texttt{1.454790e-01} & \texttt{1.690906e-01} & \texttt{1.924270e-01} \\
$10^{-2}$
 & \texttt{4.329360e-02} & \texttt{6.415001e-02} & \texttt{7.433867e-02} & \texttt{8.435152e-02} \\
 & \texttt{3.264508e-01} & \texttt{4.796196e-01} & \texttt{5.534001e-01} & \texttt{6.252452e-01} \\
\hline
 & \texttt{9.351562e-03} & \texttt{1.401039e-02} & \texttt{1.633542e-02} & \texttt{1.865748e-02} \\
$10^{-3}$
 & \texttt{4.027555e-03} & \texttt{6.031164e-03} & \texttt{7.030384e-03} & \texttt{8.027867e-03} \\
 & \texttt{2.960255e-02} & \texttt{4.431587e-02} & \texttt{5.165044e-02} & \texttt{5.897025e-02} \\
\hline
\end{tabular}
\caption{Ошибки для схемы 1 при $C = 1$, $\gamma = 1.4$ и~$\mu = 0.1$.}
\end{table}

\begin{table}[H]
\centering
\begin{tabular}{|c|c|c|c|c|}
\hline
\diagTHk & $1/2$ & $1/4$ & $1/8$ & $u$ \\
\hline
 & \texttt{1.784096e+00} & \texttt{1.859894e+00} & \texttt{1.870118e+00} & \texttt{1.866822e+00} \\
$10^{-1}$
 & \texttt{7.849234e-01} & \texttt{8.500746e-01} & \texttt{8.757011e-01} & \texttt{9.253075e-01} \\
 & \texttt{9.106497e+00} & \texttt{1.051391e+01} & \texttt{1.099866e+01} & \texttt{1.105338e+01} \\
\hline
 & \texttt{1.077708e-01} & \texttt{1.533210e-01} & \texttt{1.856236e-01} & \texttt{2.203176e-01} \\
$10^{-2}$
 & \texttt{4.280437e-02} & \texttt{6.604274e-02} & \texttt{7.821540e-02} & \texttt{9.079278e-02} \\
 & \texttt{6.212892e-01} & \texttt{9.358895e-01} & \texttt{1.085082e+00} & \texttt{1.243512e+00} \\
\hline
 & \texttt{1.379588e-02} & \texttt{2.084938e-02} & \texttt{2.441586e-02} & \texttt{2.800903e-02} \\
$10^{-3}$
 & \texttt{5.101919e-03} & \texttt{7.681878e-03} & \texttt{8.979203e-03} & \texttt{1.028146e-02} \\
 & \texttt{8.551513e-02} & \texttt{1.294734e-01} & \texttt{1.518115e-01} & \texttt{1.743950e-01} \\
\hline
\end{tabular}
\caption{Ошибки для схемы 1 при $C = 1$, $\gamma = 1$ и~$\mu = 0.01$.}
\end{table}

\begin{table}[H]
\centering
\begin{tabular}{|c|c|c|c|c|}
\hline
\diagTHk & $1/2$ & $1/4$ & $1/8$ & $u$ \\
\hline
 & \texttt{3.124269e+01} & \texttt{3.094572e+01} & \texttt{3.073160e+01} & \texttt{3.042338e+01} \\
$10^{-1}$
 & \texttt{1.794478e+01} & \texttt{1.788934e+01} & \texttt{1.787415e+01} & \texttt{1.786137e+01} \\
 & \texttt{1.135489e+02} & \texttt{1.105171e+02} & \texttt{1.090479e+02} & \texttt{1.071383e+02} \\
\hline
 & \texttt{1.675180e-01} & \texttt{2.594057e-01} & \texttt{3.081415e-01} & \texttt{3.587676e-01} \\
$10^{-2}$
 & \texttt{6.249275e-02} & \texttt{9.796446e-02} & \texttt{1.171006e-01} & \texttt{1.372684e-01} \\
 & \texttt{4.993892e-01} & \texttt{7.640619e-01} & \texttt{9.019860e-01} & \texttt{1.042954e+00} \\
\hline
 & \texttt{2.020218e-02} & \texttt{3.040301e-02} & \texttt{3.552424e-02} & \texttt{4.065698e-02} \\
$10^{-3}$
 & \texttt{8.038578e-03} & \texttt{1.212124e-02} & \texttt{1.417807e-02} & \texttt{1.624499e-02} \\
 & \texttt{5.664685e-02} & \texttt{8.509400e-02} & \texttt{9.932958e-02} & \texttt{1.135627e-01} \\
\hline
\end{tabular}
\caption{Ошибки для схемы 1 при $C = 10$, $\gamma = 1$ и~$\mu = 0.01$.}
\end{table}

\begin{table}[H]
\centering
\begin{tabular}{|c|c|c|c|c|}
\hline
\diagTHk & $1/2$ & $1/4$ & $1/8$ & $u$ \\
\hline
 & \texttt{1.176241e+07} & \texttt{2.219469e+04} & \texttt{4.060771e+04} & \texttt{2.276056e+01} \\
$10^{-1}$
 & \texttt{9.864888e+06} & \texttt{1.212547e+04} & \texttt{2.387947e+04} & \texttt{1.285530e+01} \\
 & \texttt{4.997798e+07} & \texttt{4.801224e+04} & \texttt{1.374986e+05} & \texttt{1.250047e+02} \\
\hline
 & \texttt{4.805052e-01} & \texttt{3.882553e-01} & \texttt{4.093621e-01} & \texttt{3.772718e-01} \\
$10^{-2}$
 & \texttt{2.240329e-01} & \texttt{1.923434e-01} & \texttt{1.907162e-01} & \texttt{1.893409e-01} \\
 & \texttt{2.440707e+01} & \texttt{1.429881e+01} & \texttt{1.318739e+01} & \texttt{1.284382e+01} \\
\hline
 & \texttt{3.114381e-04} & \texttt{4.661482e-04} & \texttt{5.471824e-04} & \texttt{6.438681e-04} \\
$10^{-3}$
 & \texttt{2.050992e-04} & \texttt{2.805874e-04} & \texttt{3.139441e-04} & \texttt{3.545313e-04} \\
 & \texttt{1.467146e-03} & \texttt{2.234530e-03} & \texttt{2.675030e-03} & \texttt{3.333160e-03} \\
\hline
\end{tabular}
\caption{Ошибки для схемы 1 при $C = 100$, $\gamma = 1$ и~$\mu = 0.01$.}
\end{table}

\begin{table}[H]
\centering
\begin{tabular}{|c|c|c|c|c|}
\hline
\diagTHk & $1/2$ & $1/4$ & $1/8$ & $u$ \\
\hline
 & \texttt{6.567249e+00} & \texttt{6.696712e+00} & \texttt{6.721134e+00} & \texttt{6.755426e+00} \\
$10^{-1}$
 & \texttt{3.032587e+00} & \texttt{3.190672e+00} & \texttt{3.235410e+00} & \texttt{3.255147e+00} \\
 & \texttt{2.601931e+01} & \texttt{2.560890e+01} & \texttt{2.551919e+01} & \texttt{2.570541e+01} \\
\hline
 & \texttt{7.391739e-02} & \texttt{1.167851e-01} & \texttt{1.397961e-01} & \texttt{1.638207e-01} \\
$10^{-2}$
 & \texttt{3.197446e-02} & \texttt{4.987383e-02} & \texttt{5.952742e-02} & \texttt{6.974711e-02} \\
 & \texttt{4.404070e-01} & \texttt{6.890184e-01} & \texttt{8.194399e-01} & \texttt{9.778844e-01} \\
\hline
 & \texttt{1.016908e-02} & \texttt{1.536678e-02} & \texttt{1.799479e-02} & \texttt{2.064166e-02} \\
$10^{-3}$
 & \texttt{4.262132e-03} & \texttt{6.441050e-03} & \texttt{7.542957e-03} & \texttt{8.653362e-03} \\
 & \texttt{6.751022e-02} & \texttt{1.022779e-01} & \texttt{1.199909e-01} & \texttt{1.380703e-01} \\
\hline
\end{tabular}
\caption{Ошибки для схемы 1 при $C = 1$, $\gamma = 1.4$ и~$\mu = 0.01$.}
\end{table}

\begin{table}[H]
\centering
\begin{tabular}{|c|c|c|c|c|}
\hline
\diagTHk & $1/2$ & $1/4$ & $1/8$ & $u$ \\
\hline
 & \texttt{2.366081e+00} & \texttt{2.505394e+00} & \texttt{2.515007e+00} & \texttt{2.499247e+00} \\
$10^{-1}$
 & \texttt{8.871389e-01} & \texttt{9.832300e-01} & \texttt{1.042422e+00} & \texttt{1.110041e+00} \\
 & \texttt{1.151615e+01} & \texttt{1.291425e+01} & \texttt{1.349871e+01} & \texttt{1.324640e+01} \\
\hline
 & \texttt{1.958573e-01} & \texttt{2.544976e-01} & \texttt{2.771897e-01} & \texttt{2.979725e-01} \\
$10^{-2}$
 & \texttt{5.782859e-02} & \texttt{8.683485e-02} & \texttt{1.016636e-01} & \texttt{1.168625e-01} \\
 & \texttt{1.305147e+00} & \texttt{1.638417e+00} & \texttt{1.786891e+00} & \texttt{1.909283e+00} \\
\hline
 & \texttt{1.973661e-02} & \texttt{3.028482e-02} & \texttt{3.574973e-02} & \texttt{4.131314e-02} \\
$10^{-3}$
 & \texttt{6.449522e-03} & \texttt{9.715364e-03} & \texttt{1.135914e-02} & \texttt{1.301039e-02} \\
 & \texttt{1.835135e-01} & \texttt{2.812976e-01} & \texttt{3.325543e-01} & \texttt{3.847270e-01} \\
\hline
\end{tabular}
\caption{Ошибки для схемы 1 при $C = 1$, $\gamma = 1$ и~$\mu = 0.001$.}
\end{table}

\begin{table}[H]
\centering
\begin{tabular}{|c|c|c|c|c|}
\hline
\diagTHk & $1/2$ & $1/4$ & $1/8$ & $u$ \\
\hline
 & \texttt{2.198164e+01} & \texttt{2.202894e+01} & \texttt{2.201098e+01} & \texttt{2.187930e+01} \\
$10^{-1}$
 & \texttt{1.224404e+01} & \texttt{1.217412e+01} & \texttt{1.210386e+01} & \texttt{1.196209e+01} \\
 & \texttt{1.150774e+02} & \texttt{1.151122e+02} & \texttt{1.146423e+02} & \texttt{1.139058e+02} \\
\hline
 & \texttt{3.849609e-01} & \texttt{4.284301e-01} & \texttt{4.283033e-01} & \texttt{4.283601e-01} \\
$10^{-2}$
 & \texttt{9.035976e-02} & \texttt{1.221591e-01} & \texttt{1.402430e-01} & \texttt{1.600718e-01} \\
 & \texttt{8.356215e+00} & \texttt{8.502538e+00} & \texttt{8.519629e+00} & \texttt{8.536486e+00} \\
\hline
 & \texttt{2.214952e-02} & \texttt{3.330891e-02} & \texttt{3.890181e-02} & \texttt{4.449849e-02} \\
$10^{-3}$
 & \texttt{8.822492e-03} & \texttt{1.330005e-02} & \texttt{1.555455e-02} & \texttt{1.781895e-02} \\
 & \texttt{6.206949e-02} & \texttt{9.313198e-02} & \texttt{1.086351e-01} & \texttt{1.240966e-01} \\
\hline
\end{tabular}
\caption{Ошибки для схемы 1 при $C = 10$, $\gamma = 1$ и~$\mu = 0.001$.}
\end{table}

\begin{table}[H]
\centering
\begin{tabular}{|c|c|c|c|c|}
\hline
\diagTHk & $1/2$ & $1/4$ & $1/8$ & $u$ \\
\hline
 & \texttt{8.179748e+04} & \texttt{1.279995e+04} & \texttt{1.455549e+04} & \texttt{1.146038e+02} \\
$10^{-1}$
 & \texttt{3.035173e+04} & \texttt{1.034660e+04} & \texttt{5.001999e+03} & \texttt{7.974051e+01} \\
 & \texttt{3.328203e+05} & \texttt{5.311025e+04} & \texttt{5.687223e+04} & \texttt{3.096951e+02} \\
\hline
 & \texttt{4.502753e+04} & \texttt{3.720091e+06} & \texttt{4.276134e+04} & \texttt{4.276125e+04} \\
$10^{-2}$
 & \texttt{2.369080e+04} & \texttt{2.165964e+06} & \texttt{2.184381e+04} & \texttt{2.184386e+04} \\
 & \texttt{3.594172e+05} & \texttt{7.924567e+06} & \texttt{3.253861e+05} & \texttt{3.253859e+05} \\
\hline
 & \texttt{3.659287e-01} & \texttt{3.190817e-01} & \texttt{3.195958e-01} & \texttt{3.195484e-01} \\
$10^{-3}$
 & \texttt{1.321617e-01} & \texttt{1.533102e-01} & \texttt{1.530107e-01} & \texttt{1.530316e-01} \\
 & \texttt{1.081254e+02} & \texttt{9.235750e+01} & \texttt{9.134213e+01} & \texttt{9.128367e+01} \\
\hline
\end{tabular}
\caption{Ошибки для схемы 1 при $C = 100$, $\gamma = 1$ и~$\mu = 0.001$.}
\end{table}

\begin{table}[H]
\centering
\begin{tabular}{|c|c|c|c|c|}
\hline
\diagTHk & $1/2$ & $1/4$ & $1/8$ & $u$ \\
\hline
 & \texttt{1.090468e+01} & \texttt{1.095785e+01} & \texttt{1.077425e+01} & \texttt{1.069411e+01} \\
$10^{-1}$
 & \texttt{4.262182e+00} & \texttt{4.069871e+00} & \texttt{3.953579e+00} & \texttt{3.903570e+00} \\
 & \texttt{4.458606e+01} & \texttt{4.326100e+01} & \texttt{4.219661e+01} & \texttt{4.161031e+01} \\
\hline
 & \texttt{1.631455e-01} & \texttt{2.105736e-01} & \texttt{2.528627e-01} & \texttt{3.006917e-01} \\
$10^{-2}$
 & \texttt{6.343158e-02} & \texttt{8.943870e-02} & \texttt{1.026367e-01} & \texttt{1.162158e-01} \\
 & \texttt{1.046706e+00} & \texttt{1.261443e+00} & \texttt{1.409836e+00} & \texttt{1.598208e+00} \\
\hline
 & \texttt{1.887220e-02} & \texttt{2.867918e-02} & \texttt{3.368103e-02} & \texttt{3.875467e-02} \\
$10^{-3}$
 & \texttt{6.031409e-03} & \texttt{9.079104e-03} & \texttt{1.061075e-02} & \texttt{1.214773e-02} \\
 & \texttt{1.035952e-01} & \texttt{1.570542e-01} & \texttt{1.842068e-01} & \texttt{2.116635e-01} \\
\hline
\end{tabular}
\caption{Ошибки для схемы 1 при $C = 1$, $\gamma = 1.4$ и~$\mu = 0.001$.}
\end{table}



\subsection{Вывод}
Исходя из полученных результатов, можно сделать вывод, что разностная схема сходится в зависимости от внешних параметров. При больших $\mu$ и малых $C$ сходимость безусловная, а при уменьшении $\mu$ или увеличении $C$ появляется условие сходимости $\tau < h$. Сходимость имеет порядок $\tau + h^2$.
