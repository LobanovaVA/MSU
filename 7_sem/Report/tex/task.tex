\section{Описание схемы}

\subsection{Схема}
Обозначим за $G$ и $V$ приближенные значения функций $\ln \rho$ и $u$ соответственно.
Для поиска численного решения задачи (\ref{system:3}) с начальными условиями (\ref{system:2}) можно использовать разностную схему, в которой для приближения конвективных членов используются центральные разности.
Функции $G$ и $V$ на верхнем слое ищутся путем решения системы:
\begin{equation}
\label{system:4}
\left\{
 \begin{aligned}
  & 
    G_t + 0.5 [
      V \hat G_{\mathring{x}} + (V \hat G)_{\mathring{x}} +
      2 \hat V_{\mathring{x}} - G V_{\mathring{x}} ] = 0 
  \\
  & 
    G_{t,0} + 0.5 [
      (V \hat G)_{x,0} + 2 \hat V_{x,0} - G_0 V_{x,0}] - \\
  & - 0.5h [
      (GV)_{x \bar x, 1} - 0.5(GV)_{x \bar x, 2} +
      (2 - G_0) (V_{x \bar x, 1} - 0.5V_{x \bar x, 2}) ] = 0
  \\
  & 
    G_{t,M} + 0.5 [
      (V \hat G)_{\bar x,M} + 2 \hat V_{\bar x,M} - G_0 V_{\bar x,M}] + \\
  & + 0.5h [
      (GV)_{x \bar x, M-1} - 0.5(GV)_{x \bar x, M-2} +
      (2 - G_M) (V_{x \bar x, M-1} - 0.5V_{x \bar x, M-2}) ] = 0
  \\
  & 
    V_t + \frac13 [V \hat V_{\mathring{x}} + (V \hat V)_{\mathring{x}} ] + 
    \tilde{p}'(e^G) \hat G_{\mathring{x}} =
    \tilde{\mu} \hat V_{x \bar x} - 
    (\tilde{\mu} - \mu e^{-G}) V_{x \bar x} + f
 \end{aligned}
\right.
\end{equation}

где $ \tilde{\mu} = \mu \lVert e^{-G} \rVert_C $ и 
$\tilde{p}'(x) = \dfrac{\partial p}{\partial \rho}(x)$

В качестве значений разностного решения на нулевом слое берутся проекции на
сетку $\bar w_h$ функций $\ln(\rho_0)$ и $u_0$:
$$
  G_{m}^{0} = \ln ((\rho_0)_m), \quad
  V_{m}^{0} = (u_0)_m, \quad
  m = 0,\dots,M
$$

Граничные значения функции скорости равны нулю:
$$
   V_{0}^{n} = V_{M}^{n} = 0, \quad
   n = 1,\dots,N
$$

\subsection{Координатная запись уравнений}
Пользуясь обозначениями, приведенными в разделе \ref{obozn}, перепишем уравнения из (\ref{system:4}) в координатном виде.

\subsubsection*{Первое уравнение}
$m = 1,\dots, M - 1$
\begin{equation*}
  G_t + 0.5 [
    V \hat G_{\mathring{x}} + (V \hat G)_{\mathring{x}} +
    2 \hat V_{\mathring{x}} - G V_{\mathring{x}} ] = 0 
\end{equation*}
\begin{multline*}
  \frac{G_{m}^{n+1} - G_{m}^{n}}{\tau} + \frac12 \left[
    V_{m}^{n} \frac{G_{m+1}^{n+1} - G_{m-1}^{n+1}}{2h} +
    \frac{V_{m+1}^{n} G_{m+1}^{n+1} - V_{m-1}^{n} G_{m-1}^{n+1}}{2h} 
    \right. +{} \\ {}+ \left.
    2 \frac{V_{m+1}^{n+1} - V_{m-1}^{n+1}}{2h} -
    G_{m}^{n}\frac{V_{m+1}^{n} - V_{m-1}^{n}}{2h}
  \right] = 0
\end{multline*}
\begin{multline*}
  G_{m+1}^{n+1} \left( \frac{V_{m}^{n} + V_{m+1}^{n}}{4h} \right) +
  G_{m}^{n+1}   \left( \frac{1}{\tau} \right) +
  G_{m-1}^{n+1} \left(-\frac{V_{m}^{n} + V_{m-1}^{n}}{4h} \right)
  +{} \\ {}+
  V_{m+1}^{n+1} \left( \frac{1}{2h} \right) +
  V_{m-1}^{n+1} \left(-\frac{1}{2h} \right) =
  G_{m}^{n} \left( \frac{1}{\tau} + \frac{V_{m+1}^{n} - V_{m-1}^{n}}{4h} \right) 
\end{multline*}


\paragraph{Второе уравнение}
\begin{multline*}
m = 0 
\\
  G_{t,0} + 0.5 [(V \hat G)_{x,0} + 2 \hat V_{x,0} - G_0 V_{x,0}]
  -{} \\ {}-
  0.5h [
    (GV)_{x \bar x, 1} - 0.5(GV)_{x \bar x, 2} +
    (2 - G_0) (V_{x \bar x, 1} - 0.5V_{x \bar x, 2}) ] = 0
\\
  G_{t,0} + 0.5 [(V \hat G)_{x,0} + 2 \hat V_{x,0} - G_0 V_{x,0}]
  -{} \\ {}-
  0.5h  [(GV)_{x \bar x, 1} + (2 - G_0) V_{x \bar x, 1}] +
  0.25h [(GV)_{x \bar x, 2} + (2 - G_0) V_{x \bar x, 2}]  = 0
\end{multline*}
\begin{multline*}
  \frac{G_{0}^{n+1} - G_{0}^{n}}{\tau} 
  + \frac12 \left[
    \frac{V_{1}^{n} G_{1}^{n+1} - V_{0}^{n} G_{0}^{n+1}}{h} 
    + 2 \frac{V_{1}^{n+1} - V_{0}^{n+1}}{h} 
    - G_{0}^{n} \frac{V_{1}^{n} - V_{0}^{n}}{h} 
    \right] -{} \\ {}-
 \frac{h}{2} \left[
    \frac{G_{0}^{n} V_{0}^{n} - 2 G_{1}^{n} V_{1}^{n} + G_{2}^{n} V_{2}^{n}}{h^2} + 
    (2 - G_{0}^{n}) \frac{V_{0}^{n} - 2 V_{1}^{n} + V_{2}^{n}}{h^2}
    \right] +{} \\ {}+
 \frac{h}{4} \left[
    \frac{G_{1}^{n} V_{1}^{n} - 2 G_{2}^{n} V_{2}^{n} + G_{3}^{n} V_{3}^{n}}{h^2} + 
    (2 - G_{0}^{n}) \frac{V_{1}^{n} - 2 V_{2}^{n} + V_{3}^{n}}{h^2}
    \right] = 0
\end{multline*}
\begin{equation*}
  \frac{G_{0}^{n+1} - G_{0}^{n}}{\tau} 
  + \frac12 \left[
    \frac{V_{1}^{n} G_{1}^{n+1} - V_{0}^{n} G_{0}^{n+1}}{h} 
    + 2 \frac{V_{1}^{n+1} - V_{0}^{n+1}}{h} \right] + A_{0}^{n} = 0
\end{equation*}
\begin{equation*}
  G_{1}^{n+1} \left( \frac{V_{1}^{n}}{2h} \right) +
  G_{0}^{n+1} \left( \frac{1}{\tau} - \frac{V_{0}^{n}}{2h} \right) +
  V_{1}^{n+1} \left( \frac{1}{h} \right) +
  V_{0}^{n+1} \left(-\frac{1}{h} \right) = \frac{G_{0}^{n}}{\tau} - A_{0}^{n}
\end{equation*}


\paragraph{Третье уравнение}
\begin{multline*}
m = M 
\\
  G_{t,M} + 0.5 [(V \hat G)_{\bar x,M} + 2 \hat V_{\bar x,M} - G_0 V_{\bar x,M}]
  +{} \\ {}+
  0.5h [
    (GV)_{x \bar x, M-1} - 0.5(GV)_{x \bar x, M-2} +
    (2 - G_M) (V_{x \bar x, M-1} - 0.5V_{x \bar x, M-2}) ] = 0
\\
  G_{t,M} + 0.5 [(V \hat G)_{\bar x,M} + 2 \hat V_{\bar x,M} - G_0 V_{\bar x,M}]
  +{} \\ {}+
  0.5h  [(GV)_{x \bar x, M-1} + (2 - G_M) V_{x \bar x, M-1}] -
  0.25h [(GV)_{x \bar x, M-2} + (2 - G_M) V_{x \bar x, M-2}]  = 0
\end{multline*}
\begin{multline*}
  \frac{G_{M}^{n+1} - G_{M}^{n}}{\tau} 
  + \frac12 \left[
    \frac{V_{M}^{n} G_{M}^{n+1} - V_{M-1}^{n} G_{M-1}^{n+1}}{h} 
    + 2 \frac{V_{M}^{n+1} - V_{M-1}^{n+1}}{h} 
    - G_{M}^{n} \frac{V_{M}^{n} - V_{M-1}^{n}}{h} 
    \right] +{} \\ {}+
 \frac{h}{2} \left[
    \frac{G_{M-2}^{n} V_{M-2}^{n} - 2 G_{M-1}^{n} V_{M-1}^{n} + G_{M}^{n} V_{M}^{n}}{h^2} + 
    (2 - G_{M}^{n}) \frac{V_{M-2}^{n} - 2 V_{M-1}^{n} + V_{M}^{n}}{h^2}
    \right] -{} \\ {}-
 \frac{h}{4} \left[
    \frac{G_{M-3}^{n} V_{M-3}^{n} - 2 G_{M-2}^{n} V_{M-2}^{n} + G_{M-1}^{n} V_{M-1}^{n}}{h^2} + 
    (2 - G_{M}^{n}) \frac{V_{M-3}^{n} - 2 V_{M-2}^{n} + V_{M-1}^{n}}{h^2}
    \right] = 0
\end{multline*}
\begin{equation*}
  \frac{G_{M}^{n+1} - G_{M}^{n}}{\tau} 
  + \frac12 \left[
    \frac{V_{M}^{n} G_{M}^{n+1} - V_{M-1}^{n} G_{M-1}^{n+1}}{h} 
    + 2 \frac{V_{M}^{n+1} - V_{M-1}^{n+1}}{h} \right] + A_{M}^{n} = 0
\end{equation*}
\begin{equation*}
  G_{M}^{n+1} \left( \frac{1}{\tau} - \frac{V_{M}^{n}}{2h} \right) +
  G_{M-1}^{n+1} \left( \frac{V_{M-1}^{n}}{2h} \right) +
  V_{M}^{n+1} \left( \frac{1}{h} \right) +
  V_{M-1}^{n+1} \left(-\frac{1}{h} \right) = \frac{G_{M}^{n}}{\tau} - A_{M}^{n}
\end{equation*}


\subsubsection*{Четвёртое уравнение}
$m = 1,\dots, M-1$
\begin{equation*}
  V_t + \frac13 [V \hat V_{\mathring{x}} + (V \hat V)_{\mathring{x}} ] + 
    \tilde{p}'(e^G) \hat G_{\mathring{x}} =
    \tilde{\mu} \hat V_{x \bar x} - 
    (\tilde{\mu} - \mu e^{-G}) V_{x \bar x} + f
\end{equation*}
\begin{multline*}
  \frac{V_{m}^{n+1} - V_{m}^{n}}{\tau} + \frac13 \left[
    V_{m}^{n} \frac{V_{m+1}^{n+1} - V_{m-1}^{n+1}}{2h} +
    \frac{V_{m+1}^{n} V_{m+1}^{n+1} - V_{m-1}^{n} V_{m-1}^{n+1}}{2h} 
    \right] +{} \\ {}+
  \tilde{p}'(exp(G_{m}^{n})) \frac{G_{m+1}^{n+1} - G_{m-1}^{n+1}}{2h} =
  \tilde{\mu} \frac{V_{m-1}^{n+1} - 2 V_{m}^{n+1} + V_{m+1}^{n+1}}{h^2}
  -{} \\ {}-
  (\tilde{\mu} - \mu\,exp(-G_{m}^{n}))
    \frac{V_{m-1}^{n} - 2 V_{m}^{n} + V_{m+1}^{n}}{h^2} + f_{m}^{n}
\end{multline*}
\begin{multline*}
  G_{m+1}^{n+1} \left( \frac{\tilde{p}'(exp(G_{m}^{n}))}{2h} \right) +
  G_{m-1}^{n+1} \left(-\frac{\tilde{p}'(exp(G_{m}^{n}))}{4h} \right) +
  V_{m}^{n+1} \left( \frac{1}{\tau} + \frac{2\tilde{\mu}}{h^2} \right)
  +{} \\ {}+
  V_{m+1}^{n+1} \left( 
    \frac{V_{m}^{n}}{6h} + \frac{V_{m+1}^{n}}{6h} - \frac{\tilde{\mu}}{h^2} \right) +
  V_{m-1}^{n+1} \left( 
    -\frac{V_{m}^{n}}{6h} - \frac{V_{m-1}^{n}}{6h} - \frac{\tilde{\mu}}{h^2} \right)
  = \\ =
  \frac{V_{m}^{n}}{\tau} -
  (\tilde{\mu} - \mu\,exp(-G_{m}^{n}))
    \frac{V_{m-1}^{n} - 2 V_{m}^{n} + V_{m+1}^{n}}{h^2} + f_{m}^{n}
\end{multline*}

