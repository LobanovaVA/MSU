\section{Задача о стабилизации осциллирующей функции}

\subsection{Постановка задачи}
Рассмотрим область $\Omega_x = [0,1]$. Для системы (\ref{system:1}) зададим 2 задачи, начальные и граничные условия которых определяются следующим образом:
\begin{equation}
\label{oscillating:1}
\begin{aligned}
  & \rho(0, x) = 2 + \sin (K \pi x), \; x \in [0; 1] \\
  & u(0, x) = 0, \; x \in [0; 1] \\
  & u(t, 0) = u(t, 1) = 0, \; t \in [0; T] 
\end{aligned}
\end{equation}
и
\begin{equation}
\label{oscillating:2}
\begin{aligned}
  & \rho(0, x)= 1, \; x \in [0; 1] \\
  & u(0, x) = \sin (K \pi x), \; x \in [0; 1] \\
  & u(t, 0) = u(t, 10) = 0, \; t \in [0; T] 
\end{aligned}
\end{equation}

Где $K$ - некоторое натуральное число. Для численных экспериментов желательно брать $10 K h \leq 1$. 
Также для обеих задач положим $f$ и $f_0$ из правой части (\ref{system:3}) тождественно равными нулю. \\

Суть эксперимента состоит в решении задач (\ref{oscillating:1}) и (\ref{oscillating:2}), причем вычисления следует проводить до момента времени $T = N_0 \tau$, при котором решение перестанет зависеть от времени (выйдет на стационар). 

Критерием выхода на стационар будем считать
$$
  \left\| V^{N_{0}} \right\|_C =
  \max_{m = 0 \dots M} \left| V_{m}^{N_{0}} \right| 
  \leq \varepsilon
$$
где величина $\varepsilon$ является достаточно малой и определяется опытным путем. \\


\newpage
\subsection{Численные эксперименты}
Рассматриваются задачи (\ref{oscillating:1}) и (\ref{oscillating:2}).
Зафиксируем $$ \tau = 10^{-3}, \; h = 10^{-2}, \; \varepsilon = 10^{-3}$$

Будем рассматривать зависимости 
$ p(\rho) = C \rho $ и $p(\rho) = \rho^{\gamma}$
и внешние параметры 
$$ (C, \mu) \in \{1, 10, 100\} \times \{0.1, 0.01, 0.001\}, \; \gamma = 1.4$$

Далее приведены таблицы зависимости времени стабилизации $N_0 \tau$ от параметров $\mu$, $C$, $K$ для обеих задач.
\begin{table}[H]
\centering
\begin{tabular}{|c|c|c|c|c|}
\hline
$K$ & $C = 1$     & $C = 10$    & $C = 100$   & $\gamma = 1.4$ \\
\hline
\begin{tabular}{c}
  \texttt{1} \\  \texttt{2} \\  \texttt{3} \\  \texttt{4} \\  \texttt{5}  \\ 
  \texttt{6} \\  \texttt{7} \\  \texttt{8} \\  \texttt{9} \\  \texttt{10} \\ 
\end{tabular}
&
\begin{tabular}{c}
  \texttt{2.520} \\ \texttt{11.039} \\ \texttt{2.548} \\ \texttt{8.023} \\ \texttt{1.511} \\
  \texttt{6.018} \\ \texttt{1.004}  \\ \texttt{5.013} \\ \texttt{1.001} \\ \texttt{4.012} \\
\end{tabular} 
&
\begin{tabular}{c}
  \texttt{2.854} \\ \texttt{11.083} \\ \texttt{3.025} \\ \texttt{9.179} \\ \texttt{2.214} \\
  \texttt{8.228} \\ \texttt{1.896}  \\ \texttt{7.278} \\ \texttt{1.738} \\ \texttt{6.645} \\
\end{tabular}
&
\begin{tabular}{c}
  \texttt{2.799} \\ \texttt{11.796} \\ \texttt{2.923} \\ \texttt{8.847} \\ \texttt{1.092} \\ 
  \texttt{7.372} \\ \texttt{0.727}  \\ \texttt{6.632} \\ \texttt{0.727} \\ \texttt{5.894} \\
\end{tabular}
&
\begin{tabular}{c}
  \texttt{1.254} \\ \texttt{5.213} \\ \texttt{1.511} \\ \texttt{4.606} \\ \texttt{1.302} \\ 
  \texttt{4.204} \\ \texttt{1.000} \\ \texttt{3.903} \\ \texttt{0.950} \\ \texttt{3.902} \\
\end{tabular} \\
\hline
\end{tabular}
\caption{Время стабилизации первой задачи при  $\mu = 0.1$.}
\end{table}


\begin{table}[H]
\centering
\begin{tabular}{|c|c|c|c|c|}
\hline
$K$ & $C = 1$     & $C = 10$    & $C = 100$   & $\gamma = 1.4$ \\
\hline
\begin{tabular}{c}
  \texttt{1} \\  \texttt{2} \\  \texttt{3} \\  \texttt{4} \\  \texttt{5}  \\ 
  \texttt{6} \\  \texttt{7} \\  \texttt{8} \\  \texttt{9} \\  \texttt{10} \\ 
\end{tabular}
&
\begin{tabular}{c}
  \texttt{23.032} \\ \texttt{72.039} \\ \texttt{22.081} \\ \texttt{61.022} \\ \texttt{19.974} \\
  \texttt{59.011} \\ \texttt{16.957} \\ \texttt{57.009} \\ \texttt{14.958} \\ \texttt{54.009} \\
\end{tabular}
&
\begin{tabular}{c}
  \texttt{9.978}  \\ \texttt{35.145} \\ \texttt{10.159} \\ \texttt{30.695} \\ \texttt{9.803}  \\ 
  \texttt{30.054} \\ \texttt{9.002}  \\ \texttt{29.418} \\ \texttt{8.368}  \\ \texttt{28.468} \\
\end{tabular}
&
\begin{tabular}{c}
  \texttt{1.805} \\ \texttt{7.219} \\ \texttt{2.064} \\ \texttt{6.208} \\ \texttt{1.802} \\ 
  \texttt{6.005} \\ \texttt{1.650} \\ \texttt{5.904} \\ \texttt{1.349} \\ \texttt{5.503} \\
\end{tabular}
&
\begin{tabular}{c}
  \texttt{19.530} \\ \texttt{62.782} \\ \texttt{18.862} \\ \texttt{53.172} \\ \texttt{18.202} \\ 
  \texttt{51.685} \\ \texttt{16.418} \\ \texttt{50.943} \\ \texttt{14.983} \\ \texttt{49.466} \\
\end{tabular}  \\
\hline
\end{tabular}
\caption{Время стабилизации первой задачи при  $\mu = 0.01$.}
\end{table}


\begin{table}[H]
\centering
\begin{tabular}{|c|c|c|c|c|}
\hline
$K$ & $C = 1$     & $C = 10$    & $C = 100$   & $\gamma = 1.4$ \\
\hline
\begin{tabular}{c}
  \texttt{1} \\  \texttt{2} \\  \texttt{3} \\  \texttt{4} \\  \texttt{5}  \\ 
  \texttt{6} \\  \texttt{7} \\  \texttt{8} \\  \texttt{9} \\  \texttt{10} \\ 
\end{tabular}
&
\begin{tabular}{c}
 \texttt{62.065} \\ \texttt{196.141} \\ \texttt{61.134} \\ \texttt{156.048} \\ \texttt{60.499} \\ \texttt{150.037} \\ \texttt{59.440} \\ \texttt{148.022} \\ \texttt{57.915} \\ \texttt{147.021} \\
\end{tabular}
&
\begin{tabular}{c}
 \texttt{12.986} \\ \texttt{47.182} \\ \texttt{13.331} \\ \texttt{40.824} \\ \texttt{13.445} \\ \texttt{39.547} \\ \texttt{12.797} \\ \texttt{39.225} \\ \texttt{12.002} \\ \texttt{38.905} \\
\end{tabular}
&
\begin{tabular}{c}
 \texttt{1.855} \\ \texttt{7.520} \\ \texttt{2.014} \\ \texttt{6.809} \\ \texttt{1.852} \\ \texttt{6.205} \\ \texttt{1.750} \\ \texttt{6.204} \\ \texttt{1.449} \\ \texttt{5.803} \\
\end{tabular}
&
\begin{tabular}{c}
 \texttt{39.428} \\ \texttt{138.423} \\ \texttt{41.201} \\ \texttt{112.334} \\ \texttt{41.461} \\ \texttt{107.849} \\ \texttt{41.204} \\ \texttt{106.349} \\ \texttt{40.530} \\ \texttt{106.342} \\
\end{tabular} \\
\hline
\end{tabular}
\caption{Время стабилизации первой задачи при  $\mu = 0.001$.}
\end{table}

\begin{table}[H]
\centering
\begin{tabular}{|c|c|c|c|c|}
\hline
$K$ & $C = 1$     & $C = 10$    & $C = 100$   & $\gamma = 1.4$ \\
\hline
\begin{tabular}{c}
  \texttt{1} \\  \texttt{2} \\  \texttt{3} \\  \texttt{4} \\  \texttt{5}  \\ 
  \texttt{6} \\  \texttt{7} \\  \texttt{8} \\  \texttt{9} \\  \texttt{10} \\ 
\end{tabular}
&
\begin{tabular}{c}
  \texttt{6.463} \\ \texttt{1.771} \\ \texttt{0.877} \\ \texttt{0.714} \\ \texttt{0.394} \\ 
  \texttt{0.442} \\ \texttt{0.474} \\ \texttt{0.475} \\ \texttt{0.467} \\ \texttt{0.456} \\
\end{tabular}
&
\begin{tabular}{c}
  \texttt{5.850} \\ \texttt{1.506} \\ \texttt{0.690} \\ \texttt{0.361} \\ \texttt{0.226} \\ 
  \texttt{0.135} \\ \texttt{0.118} \\ \texttt{0.106} \\ \texttt{0.097} \\ \texttt{0.090} \\
\end{tabular}
&
\begin{tabular}{c}
  \texttt{3.052} \\ \texttt{0.777} \\ \texttt{0.352} \\ \texttt{0.215} \\ \texttt{0.132} \\ 
  \texttt{0.111} \\ \texttt{0.096} \\ \texttt{0.072} \\ \texttt{0.053} \\ \texttt{0.059} \\
\end{tabular}
&
\begin{tabular}{c}
  \texttt{6.305} \\ \texttt{1.483} \\ \texttt{0.722} \\ \texttt{0.574} \\ \texttt{0.494} \\ 
  \texttt{0.285} \\ \texttt{0.317} \\ \texttt{0.339} \\ \texttt{0.344} \\ \texttt{0.340} \\
\end{tabular} \\
\hline
\end{tabular}
\caption{Время стабилизации второй задачи при $\mu = 0.1$.}
\end{table}


\begin{table}[H]
\centering
\begin{tabular}{|c|c|c|c|c|}
\hline
$K$ & $C = 1$     & $C = 10$    & $C = 100$   & $\gamma = 1.4$ \\
\hline
\begin{tabular}{c}
  \texttt{1} \\  \texttt{2} \\  \texttt{3} \\  \texttt{4} \\  \texttt{5}  \\ 
  \texttt{6} \\  \texttt{7} \\  \texttt{8} \\  \texttt{9} \\  \texttt{10} \\ 
\end{tabular}
&
\begin{tabular}{c}
  \texttt{44.391} \\ \texttt{12.705} \\ \texttt{6.145} \\ \texttt{3.615} \\ \texttt{2.498} \\ 
  \texttt{1.753}  \\ \texttt{1.364}  \\ \texttt{1.072} \\ \texttt{0.959} \\ \texttt{0.662} \\
\end{tabular}
&
\begin{tabular}{c}
  \texttt{27.362} \\ \texttt{7.676} \\ \texttt{3.538} \\ \texttt{2.023} \\ \texttt{1.367} \\ 
  \texttt{0.983}  \\ \texttt{0.662} \\ \texttt{0.501} \\ \texttt{0.411} \\ \texttt{0.372} \\
\end{tabular}
&
\begin{tabular}{c}
  \texttt{5.753} \\ \texttt{1.578} \\ \texttt{0.719} \\ \texttt{0.390} \\ \texttt{0.232} \\ 
  \texttt{0.228} \\ \texttt{0.109} \\ \texttt{0.083} \\ \texttt{0.086} \\ \texttt{0.078} \\
\end{tabular}
&
\begin{tabular}{c}
  \texttt{41.958} \\ \texttt{12.069} \\ \texttt{5.788} \\ \texttt{3.496} \\ \texttt{2.292} \\ 
  \texttt{1.630}  \\ \texttt{1.279}  \\ \texttt{0.909} \\ \texttt{0.812} \\ \texttt{0.648} \\
\end{tabular} \\
\hline
\end{tabular}
\caption{Время стабилизации второй задачи при $\mu = 0.01$.}
\end{table}



\begin{table}[H]
\centering
\begin{tabular}{|c|c|c|c|c|}
\hline
$K$ & $C = 1$     & $C = 10$    & $C = 100$   & $\gamma = 1.4$ \\
\hline
\begin{tabular}{c}
  \texttt{1} \\  \texttt{2} \\  \texttt{3} \\  \texttt{4} \\  \texttt{5}  \\ 
  \texttt{6} \\  \texttt{7} \\  \texttt{8} \\  \texttt{9} \\  \texttt{10} \\ 
\end{tabular}
&
\begin{tabular}{c}
 \texttt{154.584} \\ \texttt{46.299} \\ \texttt{30.216} \\ \texttt{26.722} \\ \texttt{17.171} \\ \texttt{9.128} \\ \texttt{6.116} \\ \texttt{3.970} \\ \texttt{8.153} \\ \texttt{4.100} \\
\end{tabular}
&
\begin{tabular}{c}
 \texttt{45.089} \\ \texttt{12.586} \\ \texttt{6.074} \\ \texttt{3.530} \\ \texttt{2.320} \\ \texttt{1.672} \\ \texttt{1.254} \\ \texttt{0.860} \\ \texttt{0.767} \\ \texttt{0.596} \\
\end{tabular}
&
\begin{tabular}{c}
 \texttt{6.353} \\ \texttt{1.728} \\ \texttt{0.652} \\ \texttt{0.415} \\ \texttt{0.313} \\ \texttt{0.194} \\ \texttt{0.138} \\ \texttt{0.147} \\ \texttt{0.120} \\ \texttt{0.088} \\
\end{tabular}
&
\begin{tabular}{c}
 \texttt{133.715} \\ \texttt{40.477} \\ \texttt{21.801} \\ \texttt{11.594} \\ \texttt{8.243} \\ \texttt{6.008} \\ \texttt{4.660} \\ \texttt{3.322} \\ \texttt{3.250} \\ \texttt{2.403} \\
\end{tabular} \\
\hline
\end{tabular}
\caption{Время стабилизации второй задачи при $\mu = 0.001$.}
\end{table}

\subsection{Вывод}
Анализируя таблицы можно сделать следующие выводы:
\begin{enumerate}
  \item Время стабилизации зависит от четности параметра $K$ : при четных значениях время заметно выше, чем при нечётных. 
С уменьшением параметра $\mu$ время стабилизации увеличивается. 
С увеличением параметра $C$   время стабилизации уменьшается. 
\item С увеличением параметра $K$ время стабилизации падает. 
С увеличением параметра $C$ время стабилизации уменьшается. 
C уменьшением параметра $\mu$  время стабилизации увеличивается значительно для малых $K$ и незначительно для больших $K$.
\end{enumerate}




