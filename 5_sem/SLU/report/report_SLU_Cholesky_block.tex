\documentclass[a4paper,12pt]{article}
\usepackage[T2A]{fontenc}
\usepackage[utf8]{inputenc}
\usepackage[english,russian]{babel}
\usepackage[dvipsnames]{xcolor}
\usepackage{amssymb}
\usepackage{amsmath}
\usepackage{amsthm}
\usepackage{enumitem}
\usepackage{cmap}
\usepackage{hyperref}
\usepackage{mdframed}
\usepackage{anyfontsize}

%\setlist{nolistsep}

\newtheorem*{task*}{Задача}
\newtheorem*{solve*}{Идея решения}
\newtheorem*{theorem*}{Теорема}

\title{\textbf{Решение системы линейных уравнений блочным методом Холецкого}}
\author{Лобанова Валерия, группа 310}
\date{ }

\begin{document}

% Обложка
\maketitle
\thispagestyle{empty} 
% Содержание
\newpage
\tableofcontents{}

% Глава1 
\newpage
\pagestyle{plain}
\section{Введение}
\subsection{Постановка задачи. Разложение Холецкого}
    \begin{task*}
        Найти решение системы линейных уравнений $A x=b$, где \\
        $A $ --- симметричная вещественнозначная матрицы размера $n \times n$, \\
        $b$ --- известный вектор размера $n$, \\
        $x$ --- неизвестный вектор.
    \end{task*}
    
    \begin{proof} [Идея решения]
        Поиск решения будет осуществляться с помощью разложения Холецкого матрицы $A=R^T DR$,  где \\
        $R$ --- верхнетреугольная матрица, \\
        $D$ --- диагональная матрица с $1$ или $-1$  на диагонали.\\
        Найдем такое $y$, что $R^Ty=b$ и затем из условия $DRx=y$ найдем $x$.
    \end{proof}
    
    \begin{theorem*}
        Пусть матрица $A$ --- самосопряженная и все ее угловые миноры отличны
        от нуля. Тогда существует матрица $R=(r_{ij}) \in RT(n)$  с вещественными 
        положительными элементами на главной диагонали и диагональная матрица
        $D$ с вещественными равными по модулю единице дигональными элементами
        такие, что $A=R^T DR$.
    \end{theorem*}
    
    \begin{proof} [Решение задачи]
        Применим точечный метод Холецкого для поиска матрицы $R$.
        Элементы $d_{ii}$, $r_{ii}$ ,$r_{ij}$ могут быть вычиcлены по 
        следующим формулам:
        \begin{equation} \label{dot}
             d_{ii} = sgn(a_{ii}-\sum_{k=1}^{i-1}|r_{ki}|^2d_{kk}),\  i=1,...,n,
        \end{equation}
        $$ 
            r_{ii} = \sqrt{\Big|a_{ii}-\sum_{k=1}^{i-1}|r_{ki}|^2d_{kk}\Big|},\  i=1,...,n, 
        $$
        $$ 
            r_{ij} = (r_{ii}d_{ii})^{-1}
                    (a_{ij}-\sum_{k=1}^{i-1}r_{ki}d_{kk}r_{kj}),\  
                    i<j, \ i,j=1,...,n,
        $$
    \end{proof}



\newpage
\subsection{Оценка сложности алгоритма построения \\ верхнетреугольной матрицы
в разложении Холецкого}

    Из формул~\eqref{dot} следует, что для вычисления элемента $d_{ii}$, $i=1,...,n$\\
    требуется $2 (i-1)$ операций (умножение на $d_{ii}$ за операцию не считаем). \\
    Следовательно, вычисление всех элементов матрицы $D$ требует \\
    $$
        \sum_{i=1}^n 2(i-1) = n(n-1) = O(n^2), \
        n \rightarrow \infty 
        \quad \text{операций.}
    $$
    
    Для вычисления элемента $r_{ii}$ требуется $2 (i-1) + 1 = 2i - 1$ операций 
    (учитываем 1 операцию извлечения корня). \\
    При фиксированном $i=1,...,n$ вычисление элементов $r_{ij}$ для всех 
    $j=i+1,..n$ по формулам~\eqref{dot} требует 
    $$\sum_{j=i+1}^n(2i - 1) = (n-i)(2i - 1) \quad \text{операций.}$$ 

    Таким образом нахождение матрицы $R$ требует 
    $$
        \sum_{i=1}^n(n-i)(2i - 1)+(2i - 1) = 
        \frac{2n^2 + 3n + 1}6  =  \frac{n^3}3 + O(n^2), \
        n \rightarrow \infty \
        \quad \text{операций.}
    $$


% Глава2
\newpage
\section{Блочный метод Холецкого}
\subsection{Описание блочного метода Холецкого}
    Разобьем матрицу $A$ на блоки $(A_{ij})$ размера $m \times m$, где $m < n$ и 
    если $m \nmid n \Rightarrow n = m*k + l, l \neq 0$, то крайние блоки могут иметь 
    размеры $m \times l$, или $l \times m$, или $l \times l$. 
    Матрицы $R$ и $D$ можно также искать в виде блочных матриц. \\
    
    Из формулы $A = R^TDR$ ясно, что формулы для нахождения блоков 
    матрицы $R$ имеют вид:
    \begin{equation} \label{block_i}
        R_{ii}^T D_i R_{ii}  = A_{ii} - 
        \sum_{j=1}^{i-1} R_{ji}^T D_j R_{ji},\  
        i=1,...,k,
     \end{equation}
    $$
        R_{ii}^T D_i R_{is}  = A_{is} - 
        \sum_{j=1}^{i-1} R_{ji}^T D_j R_{js},\  
        i, s = 1,...,k, \ i < s
    $$
    \begin{equation} \label{block_is}
        R_{is} = D_i(R_{ii}^T)^{-1}(A_{is} - 
        \sum_{j=1}^{i-1} R_{ji}^T D_j R_{js}),\  
        i, s = 1,...,k, \ i < s
    \end{equation}
    
    Тем самым сначала $R_{ii}$ и $D_i$ ищутся разложением из~\eqref{block_i}, 
    а после для $s = i+1,...n$ вычисляются $R_{is}$, используя формулу~\eqref{block_is}.


\newpage
\subsection{Оценка сложности в алгоритме построения верхнетреугольной матрицы
в блочном разложении Холецкого}   
    Если известно количество операций в случае $l = 0$, то количество 
    операций в случае $l \neq 0$ можно оценить сверху, сделав в имеющейся оценке замену $k$ на $k + 1$.  \\
    
    Начнём с оценки количества операций для $R_{ji}^T D_j R_{js}$ ,\\
    чтобы не путаться в индексах рассмотим это произвдение как $R^TDR$, тогда
    $$ (R^TDR)_{ij} = \sum_{k=1}^{min(i,j)} r_{ki} d_k r_{kj} $$
    здесь $min(i,j) - 1$ аддитивных и $min(i,j)$ мультипликативных операций, 
    то есть всего $2min(i,j) - 1$ операций для одного элемента 
    (по аналогии с неблочным методом умножение на $d_i$ за операцию не считаем). \\
    
    Тогда для вычисления $R_{ji}^T D_j R_{js}$ требуется 
    $$
        \sum_{i=1}^m \sum_{j=1}^m (2min(i,j) - 1) =
        \sum_{i=1}^m \sum_{j=1}^i (2j - 1) + 
        \sum_{i=1}^m \sum_{j=i+1}^m (2i - 1) = $$$$
        \frac{m(m+1)(2m+1)}6 + \frac{m(2m^2-3m+1)}6 = 
        \frac{m(2m^2+1)}3 
    $$
    
    Обозначим как $ Mult(m) = (2m^3+m)/3 $. \\ \\
    
    Для вычисления $A_{is} - \sum_{k=1}^{i-1}R_{ji}^T D_j R_{js}$ требуется: 
    $$
        H(i,m) = 
        (i-1)(Mult(m)+m^2) = 
        (i-1)(2m^3+3m^2+m)/3 
        \quad \text{операций.}
    $$  
    Сложность разложения Холецкого $Chol(m) = m^3/3$. \\
    Cложность вычисления блока $R_{ii}$ и $D_i$ : $H(i,m) + Chol(m)$ \\
    Сложность вычисления всех диагональных блоков:
    $$ 
        S_1(n,m,k) = 
        \sum_{i=1}^k (H(i,m) + Chol(m)) =
        n(2mn + 3n + k - 3m -1)/6
    $$
    \\
    
    Умножение на треугольную матрицу требует $Y(m)=m^3$ операций.
    Здесь имеется ввиду умножение на $(R_{ii}^T)^{-1}$ в формуле~\eqref{block_is}. 
    Подсчёт обратной к $R_{ii}^T$ учтёем позже, так как это вычисление выполняется 1 раз при подсчете всей строки. \\
    
    Итак, сложность вычисления недиагонального блока $R_{ij}$:
    $$ 
        R(i,m) = H(i,m) + Y(m) =
        (i-1)(2m^3+3m^2+m)/3 + m^3
    $$
    
    Cложность вычисления всех недиагональных блоков $R$:
    $$ 
        S_2(n,m,k) = 
        \sum_{i=1}^k\sum_{s=i+1}^k R(i,m) =
        n(k-1)(2mn+3n+k+5m^2-6m-2)/18
    $$
    
    Для вычисления строки - $R_{ij}$ при фиксированном $i$ требуется $(R_{ii}^T)^{-1}$,
    следовательно нужно $(k-1)$ раз найти обратную матрицу за \\
    $S_3(n, m) = (k-1)Chol(m) = (k-1)m^3/3$ операций. \\
    
    Итак, нахождение всех блоков $R_{is}$ требует
    $$
        S(n, m)= S_1 + S_2 + S_3 = 
        \frac{n(2mn + 3n + k - 3m -1)}6 + $$$$ +
        \frac{n(k-1)(2mn+3n+k+5m^2-6m-2)}{18} +
        \frac{(k-1)m^3}3 = $$$$ =
        \frac{n(2n^2+m^2+9mn-3m-1+3nk+k^2)}{18} - \frac{m^3}3 = $$$$ =
        \boxed {
            \frac{n^3}9 + \frac{nm^2}{18} + \frac{n^2m}3 - \frac{nm}6 -
            \frac{n}{18} + \frac{n^3}{6m} + \frac{n^3}{18m^2} - \frac{m^3}3  }
    $$
    
    $$ 
        S(n,n) =
        \frac{n^3}9 + \frac{n^3}{18} + \frac{n^3}3 - \frac{n^2}6 - 
        \frac{n}{18} + \frac{n^2}6 + \frac{n}{18} - \frac{n^3}3 = 
        \frac{n^3}3
    $$
    $$
        S(n,1) = 
        \frac{n^3}9 + \frac{n}{18} + \frac{n^2}3 - \frac{n}6 -
        \frac{n}{18} + \frac{n^3}6 + \frac{n^3}{18} - \frac{1}3 =
        \frac{n^3}3 + \frac{n^2}3 - \frac{n}6
    $$
    
    $$
        \boxed {
            S(n,n) = \frac{n^3}3 \quad 
            S(n,1) = \frac{n^3}3 + O(n^2),\ n \to \infty   }
    $$
    
    

\newpage
\subsection{Хранение матриц}
    Так как матрица $A$ симметричная, то логично хранить не всю матрицу, 
    а только верхнюю ее часть над главной диагональю и саму диагональ. 
    $$ a_{00} = a [0]; \ a_{11} = a [n]; \ a_{22} = a [n + (n-1)]; ... $$
    $$ a_{ii} = a [\,\sum_{j=0}^{i-1}(n-j)] = a [i * (2 * n - i +1) / 2];  $$
    $$ 
        a_{is} = a [\,\sum_{j=0}^{i-1}(n-j)] = a [i * (2 * n - i +1) / 2 + (s - i)], 
        \; i \leq s  
    $$
    
    У матрицы $D$ хранить нужно только диагональ в массиве длины $n$.\\
    
    При вычислении матрицы $R$ элементы $R_{is}$ можно записывать сразу на место $A_{is}$,
    так как $A_{is}$ больше не будет использоваться. 

    
\end{document}